%Este trabalho está licenciado sob a Licença Creative Commons Atribuição-CompartilhaIgual 3.0 Não Adaptada. Para ver uma cópia desta licença, visite https://creativecommons.org/licenses/by-sa/3.0/ ou envie uma carta para Creative Commons, PO Box 1866, Mountain View, CA 94042, USA.
% !TEX root = ../main.tex

\chapter{Introdução}\label{chap:introducao}

Esse livro de cálculo vetorial se dedica ao estudo de curvas, superfícies, campos vetoriais, integral de linha, integral de superfície e alguns teoremas de integração, tais como o teorema de divergência de Gauss e o teorema de Stokes, além de aplicações.

\section{Revisão}
Para uma leitura mais tranquila desse livro, é esperado que o leitor tenha os seguintes conhecimentos de Cálculo Diferencial e Integral: limites, derivadas e integrais de função de uma, duas ou três variáveis; integração em coordenadas cartesianas, polares, cilíndicas e esféricas; e outros conhecimentos mais básicos de funções elementares. Abaixo seguem alguns exercícios de revisão.

\subsection*{Exercícios Resolvidos}
\begin{exeresol}Use as expressões $\cos x=\frac{e^{ix}+e^{-ix}}{2}$ e $\sin x=\frac{e^{ix}-e^{-ix}}{2i}$ para obter as seguintes identidades trigonométricas:
\begin{itemize}
\item[a)] $\cos^2(x)+\sin^2x=1$.
\item[b)] $\cos^2(x)=\frac{1}{2}+\frac{1}{2}\cos(2x)$.
\item[c)] $\sin^2(x)=\frac{1}{2}-\frac{1}{2}\cos(2x)$.
\item[d)] $\cos^5(x)=\frac{1}{16}\left[\cos(5x)+5\cos(3x)+10\cos(x)\right]$.
\end{itemize}
\end{exeresol}
\begin{resol}
\begin{eqnarray*}
\cos^5(x)=\left(\frac{e^{ix}+e^{-ix}}{2}\right)^5=\frac{\left({e^{ix}+e^{-ix}}\right)^5}{32}
\end{eqnarray*}
Agora usamos o Binômio de Newton para $n=5$:
\begin{eqnarray*}
(a+b)^5=a^5+5a^4b+10a^3b^2+10a^2b^3+5ab^4+b^5
\end{eqnarray*}
e obtemos:
\begin{eqnarray*}
\cos^5(x)&=&\frac{{e^{5ix}+5e^{4ix}e^{-ix}+10e^{3ix}e^{-2ix}+10e^{2ix}e^{-3ix}+5e^{ix}e^{-4ix}+e^{-5ix}}}{32}\\
&=&\frac{{e^{5ix}+5e^{3ix}+10e^{ix}+10e^{-ix}+5e^{-3ix}+e^{-5ix}}}{32}\\
\end{eqnarray*}
Recoletando os termos, chegamos à resposta final, pois $\cos(5x)=\frac{e^{5ix}+e^{-5ix}}{2}$, $\cos(3x)=\frac{e^{3ix}+e^{-3ix}}{2}$ e $\cos(x)=\frac{e^{ix}+e^{-ix}}{2}$
\end{resol}

\begin{exeresol}Seja $D$ a região do plano $xy$ limitada pela circunferência de raio $1$ centrada da origem, calcule o valor da integral dada por:
$$\iint_D x^2 dxdy.$$
\end{exeresol}
\begin{resol}Parametrizando em polares, temos:
$$I= \int_0^{2\pi}\int_0^1 (\rho\cos\theta)^2\rho d\rho d\theta=\int_{0}^{2\pi}\cos^2\theta d\theta\int_0^1\rho^3d\rho=\frac{\pi}{4}$$
\end{resol}


\begin{exeresol} Seja $H$ o hemisfério limitado pela esfera unitária centrada na origem e o plano $x=0$ tal que $x>0$. Calcule o valor de $\Phi$ dado pela expressão
$$\Phi=\iiint_Hx dxdydz.$$ 
\end{exeresol}
\begin{resol}Parametrizamos a região em coordenadas esféricas:
\begin{eqnarray*}
x&=&r\sin\theta\cos\varphi,\\ y&=&r\sin\theta\sin\varphi,\qquad \\z&=&r\cos\theta,
\end{eqnarray*}
com $0\leq r \leq 1$, $0\leq \theta\leq {\pi}$, $-\frac{\pi}{2}\leq \varphi\leq \frac{\pi}{2}$.  
Portanto
\begin{eqnarray*}
\Phi&=&\int_0^1\int_0^\pi\int_{-\pi/2}^{\pi/2}r\sin\theta\cos\varphi r^2\sin\theta d\varphi d\theta dr\\
&=&\left(\int_0^1r^3dr\right)\left(\int_0^\pi\sin^2\theta d\theta \right)\left(\int_{-\pi/2}^{\pi/2}\cos\varphi d\varphi\right)\\
&=&\frac{1}{2}\left.\left(\frac{x}{2}-\frac{\sin(2x)}{4}\right)\right|_0^{\pi}=\frac{\pi}{4}
\end{eqnarray*}
\end{resol}


\subsection*{Exercícios}


\begin{exer}Use as expressões $\cosh x=\frac{e^x+e^{-x}}{2}$ e $\sinh x=\frac{e^x-e^{-x}}{2}$ para obter as seguintes identidades:
\begin{itemize}
\item[a)] $\cosh^2(x)-\sinh^2(x)=1$
\item[b)] $\cosh^2(x)=\frac{1}{2}+\frac{1}{2}\cosh(2x)$
\item[c)] $\sinh^2(x)=-\frac{1}{2}+\frac{1}{2}\cosh(2x)$
\item[d)] $e^x=\cosh(x)+\sinh(x)$
\item[e)] $e^{-x}=\cosh(x)-\sinh(x)$

\end{itemize}
\end{exer}

\begin{exer}Resolva as seguintes integrais:
\begin{itemize}
\item[a)] $\int \sin^2(x)dx$
\item[b)] $\int \cos^2(x)dx$
\item[c)] $\int \cos^3(x)dx$
\end{itemize}
\end{exer}
\begin{resp}
    \begin{itemize}
        \item[a)] $\int \sin^2(x)dx=\frac{x}{2}-\frac{1}{4}\sin(2x)+C$
        \item[b)] $\int \cos^2(x)dx=\frac{x}{2}+\frac{1}{4}\sin(2x)+C$
        \item[c)] $\int \cos^3(x)dx=\frac{1}{12}\left[\sin(3x)+9\sin(x)\right]+C$
 \end{itemize}
        
\end{resp}

\begin{exer} Explique por que $|\cos(x)-1|=1-\cos(x)$.
\end{exer}
\begin{resp}
Porque $\cos(x)$ assume valores no intervalo $[-1,1]$, portanto $\cos(x)-1$ é não positivo, isto é, $\cos(x)-1\leq 0$.
\end{resp}


\begin{exer}Calcule a área da região do plano $xy$ limitada pelos eixos ordenados, as retas $x=2$ e $y=2$ e a hipérbole $xy=1$.
\end{exer}
\begin{resp}
$1+\ln4$ 
\end{resp}



