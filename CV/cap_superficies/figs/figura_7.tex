\documentclass{standalone}\usepackage{pstricks-add}\usepackage{pstricks,pst-plot}\usepackage[dvips]{graphicx}\usepackage{pst-math}\usepackage{pst-plot}\usepackage{pst-circ}\usepackage[brazil]{babel}\usepackage[utf8]{inputenc}\usepackage[T1]{fontenc}\usepackage{amsmath}\usepackage{amssymb}\usepackage{amsthm}\usepackage{mathtools}\usepackage{pst-3dplot}\usepackage{pst-solides3d}\newcommand{\sen}{\operatorname{sen}\,}\newcommand{\senh}{\operatorname{senh}\,}\renewcommand{\sin}{\operatorname{sen}\,}\renewcommand{\sinh}{\operatorname{senh}\,}
\begin{document} 
\begin{pspicture}(-2.5,-2)(2.5,1.8)
\psset{viewpoint=20 20 10,Decran=15,lightsrc=20 10 5}
 \defFunction{hip_d_a}(u,v){u sinh v Cos mul}{u sinh v Sin mul}{u cosh 1.5 div} 
 \psSolid[object=surfaceparametree,base=-1.6 0 0 6.28,fillcolor=yellow!50,incolor=green!50,function=hip_d_a,linewidth=0.5\pslinewidth,ngrid=30 30]%
  \defFunction{hip_d_b}(u,v){u sinh v Cos mul}{u sinh v Sin mul}{u cosh -2 div} 
 \psSolid[object=surfaceparametree,base=0 1.6 0 6.28,fillcolor=yellow!50,incolor=green!50,function=hip_d_b,linewidth=0.5\pslinewidth,ngrid=30 30]%
\psset{linecolor=black}\black
\rput(0,-1.8){Hiperbolóide de duas folhas}
\axesIIID[linecolor=blue](0,0,0)(4,3,2)
 \end{pspicture}  
\end{document}