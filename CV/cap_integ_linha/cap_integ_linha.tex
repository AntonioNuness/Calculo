%Este trabalho está licenciado sob a Licença Creative Commons Atribuição-CompartilhaIgual 3.0 Não Adaptada. Para ver uma cópia desta licença, visite https://creativecommons.org/licenses/by-sa/3.0/ ou envie uma carta para Creative Commons, PO Box 1866, Mountain View, CA 94042, USA.
% !TEX root = ../main.tex

\chapter{Integral de Linha}
  Neste capítulo, estudamos a integral de linha\index{integral de linha} e teoremas relacionados.
\section{A integral de linha para um campo escalar}
Para um campo escalar $f:D\to\mathbb{R}$, $D\subset \mathbb{R}^3$, a integral de linha ao longo de uma curva suave por partes é definida por
$$
\int_{\mathcal{C}} f(\vec{r})\, ds = \int_a^b f\left(\vec{r}(t)\right) \left|\vec{r}'(t)\right| \, dt,
$$
onde $\vec{r}$ é uma parametrização para $C$ tal que $\vec{r}(a)$ e $\vec{r}(b)$ são os pontos iniciais e finais da curva.
\section{A integral de linha para um campo vetorial}
Para um campo vetorial $\vec{F}:D\to\mathbb{R}^3$, $D\subset \mathbb{R}^3$, a integral de linha ao longo de uma curva suave por partes $C$ é definida por
$$
\int_C \vec{F}(\vec{r})\cdot\,d\vec{r} = \int_a^b \vec{F}(\vec{r}(t))\cdot\vec{r}'(t)\,dt
$$
onde $\vec{r}$ é uma parametrização para $C$ tal que $\vec{r}(a)$ e $\vec{r}(b)$ são os pontos iniciais e finais da curva, que está no sentido $\vec{r}(a)\to \vec{r}(b)$.


\subsection*{Exercícios}
\begin{exer}Calcule a integral de linha $\int_C\vec{F}\cdot d\vec{r}$ para cada campo e curva dados.
\begin{itemize}
 \item[a)] $\vec{F}=(x+2y)\vec{i}+(x-y)\vec{j},$ e $C$: $x=\cos(t)$, $y=4\sin(t)$, $0\leq t\leq \frac{\pi}{4}$.
 \item[b)] $\vec{F}=(x^2+y^2)\vec{i}-x\vec{j},$ e $C$: arco da circunferência $x^2+y^2=1$ limitado ao primeiro quadrante e orientado no sentido $(1,0)$ a $(0,1)$.
 \item[c)] $\vec{F}=\sqrt{x^2+y^2+z^2}\!~\vec{i}$ e $C$ é o segmento de reta que liga o ponto $(-1, 0, 0)$ até o ponto $(1,0,0)$.
 \item[d)]  $\vec{F}= xe^z\vec{j} + z^2\vec{k}$ e $C$ é o caminho que liga o ponto $(-1,1,0)$ até o ponto $(1,1,0)$ no plano $xy$  ao longo da curva de equação $y=x^{2n}$ onde $n$ é um número natural. Interprete o caso $n=0$.
 \item[e)] O mesmo campo vetorial do item anterior e $C$ é a poligonal formada por três segmentos de reta ligando os pontos $(-1,1,0)$,  $(-1,0,0)$, $(1,0,0)$ e $(1,1,0)$, nesta sequência. Relacione com o item anterior quando $n$ é grande.
 \end{itemize}
\end{exer}
\begin{resp}
 \begin{itemize}
  \item[a)] $1-\pi$.
  \item[b)] $-1-\frac{\pi}{4}$.
  \item[c)] $1$.
  \item[d)] $\frac{4n}{2n+1}$.
  \item[e)] $2$. 
 \end{itemize}
\end{resp}
\begin{exer}
 Calcule o trabalho ao se deslocar uma partícula no campo de força $\vec{F}=3x^2\vec{i}+(2xz-y)\vec{j}+z\vec{k}$ ao longo.
 \begin{itemize}
  \item[a)] da reta que liga $(0,0,0)$ a $(2,1,3)$. 
  \item[b)] da curva $x=2t^2$, $y=t$, $z=4t^2-t$ de $t=0$ até $t=1$.
  \item[c)] da curva definida por $x=4y$, $3x^3=8z$, de $x=0$ até $x=2$.
 \end{itemize}
 \end{exer}
\begin{resp}
 \begin{itemize}
  \item[a)] $16$.
  \item[b)] $14.2$.
  \item[c)] $16$.
 \end{itemize}
\end{resp}
\begin{exer}
 Dado $\vec{F}=(x-y)\vec{i}+(x+y)\vec{j}$, calcule $\oint\vec{F}\cdot d\vec{r}$ ao longo da curva $C$ dada pela união da curva $y=x^2$, desde o ponto $(0,0)$ até $(1,1)$, da curva $y=\sqrt{x}$, desde o ponto $(1,1)$ até $(0,0)$.
\end{exer}
\begin{resp}
 $\frac{2}{3}$.
\end{resp}




\section{O Teorema Fundamental para Integral de Linha}
\begin{teo}\index{teorema fundamental para integral de linha}Seja $\vec{F}$ um campo vetorial definido numa região $R$ do espaço e $P_0(x_0,y_0,z_0)$ e $P_1(x,y,z)$ dois pontos em $R$. Se $\vec{F}$ é conservativo \index{campo!conservativo}, isto é, $\vec{F}=\vec{\nabla}\varphi$ para algum potencial $\varphi$, então
\begin{equation}\label{TeoFundCalculo}
\int_C\vec{F}\cdot d \vec{r} =\varphi(x,y,z)-\varphi(x_0,y_0,z_0),
\end{equation}
para qualquer curva $C$ suave por partes em $R$ com início em $P_0$ e extremidade em $P$.

Reciprocamente, se
$$
\int_C\vec{F}\cdot d \vec{r}
$$
é independente da curva que começa em $P_0$ e termina em $P_1$, então o campo é conservativo.
\end{teo}
\begin{proof}
Sem perda de generalidade, suponha $C$ uma curva suave parametrizada por $\vec{r}(t)=x(t)\vec{i}+y(t)\vec{j}+z(t)\vec{k}$, $t_0\leq t\leq t_1$, com início em $\vec{r}(t_0)=P_0$ e fim em $\vec{r}(t_1)=P$. Então, supondo que $\vec{F}$ é conservativo, isto é, $\vec{F}=\vec{\nabla}\varphi$, temos
\begin{eqnarray*}
\int_C\vec{F}\cdot d \vec{r} &=& \int_C\vec{\nabla}\varphi\cdot d \vec{r}\\
&=& \int_{t_0}^{t_1}\vec{\nabla}\varphi\cdot \vec{r}'(t)dt\\
&=& \int_{t_0}^{t_1}\left(\frac{\partial\varphi}{\partial x}\vec{i}+\frac{\partial\varphi}{\partial y}\vec{j}+\frac{\partial\varphi}{\partial z}\vec{k}\right)\cdot \left(\frac{\partial x}{\partial t}\vec{i}+\frac{\partial y}{\partial t}\vec{j}+\frac{\partial z}{\partial t}\vec{k}\right) dt\\
&=& \int_{t_0}^{t_1}\left(\frac{\partial\varphi}{\partial x}\frac{\partial x}{\partial t}+\frac{\partial\varphi}{\partial y}\frac{\partial y}{\partial t}+\frac{\partial\varphi}{\partial z}\frac{\partial z}{\partial t}\right) dt.
\end{eqnarray*}
Usando a regra da cadeia, temos que o termo
$$
\frac{D\varphi}{Dt}=\frac{\partial\varphi}{\partial x}\frac{\partial x}{\partial t}+\frac{\partial\varphi}{\partial y}\frac{\partial y}{\partial t}+\frac{\partial\varphi}{\partial z}\frac{\partial z}{\partial t}
$$
é a derivada total de $\varphi$ com respeito a $t$. Logo, pelo Teorema Fundamental do Cálculo, temos:
\begin{eqnarray*}
\int_C\vec{F}\cdot d \vec{r} &=&  \int_{t_0}^{t_1}\frac{D\varphi}{Dt} dt\\
&=& \varphi(x(t_1),y(t_1),z(t_1))- \varphi(x(t_0),y(t_0),z(t_0))\\
&=& \varphi(x,y,z)- \varphi(x_0,y_0,z_0).
\end{eqnarray*}
Reciprocamente, dado que a expressão \eqref{TeoFundCalculo} é válida para qualquer curva $C$ que liga dois pontos na região, definimos a função
$$
\varphi(x,y,z)=\int_C \vec{F}\cdot d\vec{r}=\int_{P_0}^{P_1} \vec{F}\cdot d\vec{r}=\int_{(x_0,y_0,z_0)}^{(x,y,z)} \vec{F}\cdot d\vec{r},
$$
onde $C$ é uma curva qualquer que começa em $P_0$ e termina em $P$. Mostraremos que $\phi$ é o potencial do campo $\vec{F}=F_1\vec{i}+F_2\vec{j}+F_3\vec{k}$, ou seja, $\vec{F}=\vec{\nabla}\varphi$. De fato, observe que
\begin{eqnarray*}
\varphi(x+\Delta x,y,z)-\varphi(x,y,z)&=&\int_{(x_0,y_0,z_0)}^{(x+\Delta x,y,z)} \vec{F}\cdot d\vec{r}-\int_{(x_0,y_0,z_0)}^{(x,y,z)} \vec{F}\cdot d\vec{r}\\
&=&\int_{(x_0,y_0,z_0)}^{(x+\Delta x,y,z)} \vec{F}\cdot d\vec{r}+\int^{(x_0,y_0,z_0)}_{(x,y,z)} \vec{F}\cdot d\vec{r}\\
&=&\int_{(x,y,z)}^{(x+\Delta x,y,z)} \vec{F}\cdot d\vec{r}.
\end{eqnarray*}
Parametrizando um caminho reto entre $(x,y,z)$ e $(x+\Delta x,y,z)$ por $\vec{r}=t\vec{i}+y\vec{j}+z\vec{k}$, $x\leq t\leq \Delta x$, temos:
\begin{eqnarray*}
\varphi(x+\Delta x,y,z)-\varphi(x,y,z)&=&\int_x^{x+\Delta x} \vec{F}\cdot \vec{r}' dt \\
&=&\int_x^{x+\Delta x} \left(F_1\vec{i}+F_2\vec{j}+F_3\vec{k}\right)\cdot \vec{i} dt\\
&=&\int_x^{x+\Delta x} F_1 (t,y,z) dt .
\end{eqnarray*}
Logo, pelo teorema do valor médio,
\begin{eqnarray*}
\frac{\varphi(x+\Delta x,y,z)-\varphi(x,y,z)}{\Delta x}&=& \frac{1}{\Delta x}\int_x^{x+\Delta x} F_1 (t,y,z) dt\\
&=& \frac{1}{\Delta x}  F_1 (\xi,y,z) , \qquad x\leq \xi\leq x+\Delta x.
\end{eqnarray*}
Portanto,
\begin{eqnarray*}
\frac{\partial \varphi}{\partial x}&=&\lim_{\Delta x\to 0}\frac{\varphi(x+\Delta x,y,z)-\varphi(x,y,z)}{\Delta x}\\
&=&\lim_{\Delta x\to 0} \frac{1}{\Delta x}  F_1 (\xi,y,z)\\
&=&F_1 (x,y,z).
\end{eqnarray*}
Analogamente, podemos demonstrar que
$$
\frac{\partial \varphi}{\partial y}=F_2(x,y,z) \qquad \text{e}\qquad \frac{\partial \varphi}{\partial z}=F_3(x,y,z),
$$
ou seja, $\vec{F}=\vec{\nabla}\varphi.$
\end{proof}

\subsection*{Exercícios}
\begin{exer}Calcule o trabalho realizado pela força conservativa $\vec{F}=xy^2\vec{i}+x^2y\vec{j}$ ao deslocar uma partícula desde $P_0(1,1)$ até $P_1(0,0)$.
\end{exer}
\begin{resp}
 $-\frac{1}{2}$.
\end{resp}


\section{Relação entre campos irrotacionais e conservativos}
\begin{teo}Seja $\vec{F}$ um campo vetorial definido numa região $R$ do espaço. Então $\vec{F}$ é conservativo \index{campo!conservativo}, isto é, $\vec{F}=\vec{\nabla}\varphi$ para algum potencial $\varphi$, se e somente se $\vec{F}$ é irrotacional, \index{campo!irrotacional}, isto é, $\vec{\nabla}\times \vec{F}=\vec{0}$.
\end{teo}
\begin{proof}Se $\vec{F}$ é conservativo, isto é, $\vec{F}=\vec{\nabla}\varphi$, então pelo item 8 da tabela de indentidades~\ref{tab_identidades_del}, $$\vec{\nabla}\times \vec{F}=\vec{\nabla}\times \vec{\nabla}\varphi=\vec{0}.$$
Reciprocamente, dado $\vec{F}=F_1\vec{i}+F_2\vec{j}+F_3\vec{k}$ com 
$$
\vec{\nabla }\times \vec{F}=\left| \begin{array}{ccc}
 \vec{i} & \vec{j} & \vec{k} \\
 \frac{\partial}{\partial x} &\frac{\partial}{\partial y} &\frac{\partial}{\partial z} \\
F_1 & F_2 & F_3
 \end{array}\right|=\vec{0},
 $$ 
temos
\begin{equation}\label{rot_zero_eq1}
 \frac{\partial F_3}{\partial y}=\frac{\partial F_2}{\partial z},
\end{equation}
\begin{equation}\label{rot_zero_eq2}
 \frac{\partial F_1}{\partial z}=\frac{\partial F_3}{\partial x}
\end{equation}
e
\begin{equation}\label{rot_zero_eq3}
 \frac{\partial F_2}{\partial x}=\frac{\partial F_1}{\partial y}.
\end{equation}
Agora, defina
\begin{equation}\label{def_pot_campo_conser}
\varphi(x,y,z)=\int_C \vec{F}\cdot d\vec{r},
\end{equation}
onde $C$ é a curva que liga os pontos $(x_0,y_0,z_0)$ ao ponto $(x,y,z)$ através das arestas do paralelepípedo de vértices $(x_0,y_0,z_0)$, $(x,y_0,z_0)$, $(x_0,y,z_0)$, $(x,y,z_0)$, $(x_0,y_0,z)$, $(x,y_0,z)$, $(x_0,y,z)$ e $(x,y,z)$. Isto é, $C$ pode ser escrito com a união de três segmentos de retas, $C=C_1 \cup C_2\cup C_3$, com as seguintes parametrizações: 
$$
C_1: \vec{r}_1(t)=t\vec{i}+y_0\vec{j}+z_0\vec{k},\qquad x_0\leq t\leq x,
$$ 
$$
C_2: \vec{r}_2(t)=x\vec{i}+t\vec{j}+z_0\vec{k},\qquad y_0\leq t\leq y
$$ 
e
$$
C_3: \vec{r}_3(t)=x\vec{i}+y\vec{j}+t\vec{k},\qquad z_0\leq t\leq z.
$$ 
Mostraremos que $\varphi$ é um potencial para o campo $\vec{F}$, isto é, $\vec{F}=\vec{\nabla}\varphi$. De fato, da equação \eqref{def_pot_campo_conser}, usamos as parametrizações das curvas $C_1$, $C_2$ e $C_3$ para obter
\begin{eqnarray}
\nonumber \varphi(x,y,z)&=&\int_{C_1} \vec{F}\cdot d\vec{r}+\int_{C_2} \vec{F}\cdot d\vec{r}+\int_{C_3} \vec{F}\cdot d\vec{r}\\
\nonumber&=&\int_{x_0}^x \vec{F}\cdot \vec{r}'_1(t)dt+ \int_{y_0}^y \vec{F}\cdot \vec{r}'_2(t)dt+\int_{z_0}^z \vec{F}\cdot \vec{r}'_3(t)dt\\
&=&\int_{x_0}^x F_1(t,y_0,z_0) d t+ \int_{y_0}^y F_2(x,t,z_0)dt+\int_{z_0}^z F_3(x,y,t)dt.\label{def_pot_campo_conser_calculado}
\end{eqnarray}
Pelo Teorema Fundamental do Cálculo, derivamos a expressão \eqref{def_pot_campo_conser_calculado} em $z$ e obtemos a última componente do campo:
\begin{equation}\label{gradiente_3_comp}
\frac{\partial \varphi}{\partial z}=F_3(x,y,z). 
\end{equation}
Agora, derivamos a expressão \eqref{def_pot_campo_conser_calculado} em $y$
\begin{eqnarray*}
\frac{\partial \varphi}{\partial y}&=&F_2(x,y,z_0)+\frac{\partial}{\partial y}\int_{z_0}^z F_3(x,y,t)dt\\
&=&F_2(x,y,z_0)+\int_{z_0}^z \frac{\partial F_3(x,y,t)}{\partial y}dt\\ 
&=&F_2(x,y,z_0)+\int_{z_0}^z \frac{\partial F_2(x,y,t)}{\partial z}dt. 
\end{eqnarray*}
onde usamos a expressão \eqref{rot_zero_eq1} na última passagem. Assim, usamos o Teorema Fundamental do Cálculo para obter
\begin{equation}\label{gradiente_2_comp}
\frac{\partial \varphi}{\partial y}=F_2(x,y,z_0)+F_2(x,y,z)-F_2(x,y,z_0)=F_2(x,y,z). 
\end{equation}
Finalmente, derivamos a expressão \eqref{def_pot_campo_conser_calculado} com respeito a $x$:
\begin{eqnarray*}
\frac{\partial \varphi}{\partial x}&=&F_1(x,y_0,z_0) d t+ \frac{\partial}{\partial x}\int_{y_0}^y F_2(x,t,z_0)dt+ \frac{\partial}{\partial x}\int_{z_0}^z F_3(x,y,t)dt\\
&=&F_1(x,y_0,z_0) d t+ \int_{y_0}^y \frac{\partial F_2(x,t,z_0)}{\partial x}dt+ \int_{z_0}^z \frac{\partial F_3(x,y,t)}{\partial x}dt\\
&=&F_1(x,y_0,z_0) d t+ \int_{y_0}^y \frac{\partial F_1(x,t,z_0)}{\partial y}dt+ \int_{z_0}^z \frac{\partial F_1(x,y,t)}{\partial z}dt,
\end{eqnarray*}
onde usamos as expressões \eqref{rot_zero_eq2} e \eqref{rot_zero_eq3} na última passagem. Assim, usamos o Teorema Fundamental do Cálculo para concluir que
\begin{eqnarray}
\nonumber \frac{\partial \varphi}{\partial x}&=&F_1(x,y_0,z_0)+F_1(x,y,z_0)-F_1(x,y_0,z_0)+F_1(x,y,z)-F_1(x,y,z_0)\\
&=&F_1(x,y,z)\label{gradiente_1_comp}
\end{eqnarray}

Portanto, das expressões \eqref{gradiente_1_comp}, \eqref{gradiente_2_comp} e \eqref{gradiente_3_comp}, concluímos que $\vec{F}=\vec{\nabla}\varphi$, ou seja, $\vec{F}$ é um campo conservativo.
\end{proof}

\subsection*{Exercícios resolvidos}
\begin{exeresol}
  Dado o campo vetorial $\vec{F}(x,y,z) = 2xy^3+(1+3x^2y^2)\vec{j}$.
  \begin{itemize}
  \item[a)] Mostre que o campo é irrotacional.
  \item[b)] Encontre uma função potencial $\varphi(x,y,z)$.
  \item[c)] Calcule o trabalho $W:=\int_C \vec{F}\cdot d\vec{r}$ ao longo de uma curva que liga o ponto $(1,4,0)$ até o ponto $(3,1,0)$.
  \item[d)] Trace algumas curvas equipotenciais de $\varphi(x,y,z)$.
  \end{itemize}
\end{exeresol}
\begin{resol}
  Para o item a), basta calcular o rotacional:
  \begin{eqnarray*}
  \vec{\nabla}\times \vec{F} &=& \left|
  \begin{array}{ccc}
  \vec{i} & \vec{j} & \vec{k} \\
  \frac{\partial}{\partial x} & \frac{\partial}{\partial y} & \frac{\partial}{\partial z}\\
  2xy^3 & 1+3x^2y^2 & 0
  \end{array}
  \right| \\
  &=& \left(0-0\right)\vec{i} + \left(0-0\right)\vec{j} + \left(6xy^2-6xy^2\right)\vec{k}\\
  &=& \vec{0}
  \end{eqnarray*}
A função potencial $\varphi$ deve satisfazer $\vec{\nabla}\varphi=\vec{F}$, isto é:
\begin{eqnarray*}
\frac{\partial \varphi}{\partial x} &=& 2xy^3 \\
\frac{\partial \varphi}{\partial y} &=& 1+x^2y^2 \\
\frac{\partial \varphi}{\partial z} &=& 0
\end{eqnarray*}
Da primeira equação, temos $\varphi(x,y,z)= x^2y^3+C(y,z)$. Diferenciando esta expressão substituindo na segunda condição, temos:
\begin{eqnarray*}
\frac{\partial }{\partial y}(x^2y^3+C(y,z)) &=& 1+x^2y^2.
\end{eqnarray*}
Isso implica $C_y(y,z)=1$, isto é, $C(y,z) = y+D(z)$. Da última expressão vemos que $D'(z)=0$ e portanto:
$$\varphi(x,y,z)=x^2y^3+y+E.$$
Para o item c), basta aplicar o teorema das integrais de linha:
\begin{eqnarray*}
W &=& \int_C \vec{F}\cdot d\vec{r} = \varphi(3,1,0)- \varphi(1,4,0)\\
  &=& (9+1+E) - (64+4+E) = -58.
\end{eqnarray*}
\end{resol}

\subsection*{Exercícios}
\begin{exer}Determine se o campo $\vec{F}$ é conservativo. Se for, encontre a função potencial $\phi$.
 \begin{itemize}
  \item[a)] $\vec{F}=x\vec{i}+y\vec{j}$.
  \item[b)] $\vec{F}=x^2y\vec{i}+5xy^2\vec{j}$.
  \item[c)] $\vec{F}=(\cos(y)+y\cos(x))\vec{i}+(\sin(x)-x\sin(y))\vec{j}$.
  \item[d)] $\vec{F}=yz\vec{i}+xz\vec{j}+(xy+3z^2)\vec{k}$.
 \end{itemize}
 \end{exer}
\begin{resp}
  \begin{itemize}
  \item[a)] $\phi=\frac{1}{2}\left(x^2+y^2\right)+C$.
  \item[b)] Não é conservativo.
  \item[c)] $\phi=x\cos(y)+y\sin(x)+C$.
  \item[d)] $\phi=xyz+z^3+C$.
 \end{itemize}
\end{resp}

\begin{exer}
 Calcule $\int_C\vec{F}\cdot d\vec{r}$, para $\vec{F}=(e^y+ye^x)\vec{i}+(xe^y+e^x)\vec{j}$ e $C$ é a curva dada por $\vec{r}=\sin\left(\frac{\pi}{2}t\right)\vec{i}+(\ln(t))\vec{j}$, $1\leq t\leq 2$. 
 
 {\it (Sugestão: Determine primeiro se $\vec{F}$ é conservativo).}
\end{exer}
\begin{resp}
 $-1+\ln(2)$.
\end{resp}

\begin{exer}Se $\vec{A}=(4xy-3x^2z^2)\vec{i}+2x^2\vec{j}-2x^3z\vec{k}$, prove que $\int_C\vec{A}\cdot d\vec{r}$ é independente da curva que liga dois pontos dados. 
\end{exer}
\begin{exer}Seja $\vec{E}=r\vec{r}$.
 \begin{itemize}
  \item[a)] Prove que existe um potencial $\phi$ tal que $\vec{E}=\vec{\nabla}\phi$.
  \item[b)] Calcule $\phi$.
  \item[c)] Calcule o valor de $\oint_C\vec{E}\cdot d\vec{r}$ para $C$ a circunferência de raio unitário no plano $xy$ orientada no sentido horário. É possível conhecer o valor de $\oint_C\vec{E}\cdot d\vec{r}$ onde $C$ é uma curva fechada qualquer? \end{itemize}
\end{exer}

 \begin{resp}
 \begin{itemize}
  \item[b)] $\phi=\frac{r^3}{3}+C$.
  \item[c)] $0$
 \end{itemize}
 \end{resp}

 \begin{exer}
  Calcule o trabalho para se deslocar uma partícula nos campos de força conservativa dados por:
  \begin{itemize}
   \item[a)] $\vec{F}=-k\vec{r}$ (Lei de Hooke).
   \item[b)] $\vec{F}=\frac{1}{r^3}\vec{r}$ (Lei do inverso do quadrado).
  \end{itemize}
 \end{exer}
\begin{resp}
   \begin{itemize}
   \item[a)] $\phi=-\frac{k r^2}{2}+C$.
   \item[b)] $\phi=-\frac{1}{r}+C$.
  \end{itemize}
\end{resp}



\section{O Teorema de Green}
\begin{teo}
 Seja $C$ uma curva simples, fechada e derivável, $D$ a região do plano delimitada por $C$, e $P$ e $Q$ duas funções reais de variável real com derivadas parciais contínua numa região contendo $D$, então:

$$\int_{C} (P dx + Q dy) = \iint_{D} \left(\frac{\partial Q}{\partial x} - \frac{\partial P}{\partial
y}\right) dA.
$$
Em termo de notação, também se costuma usar $\oint_C$ para enfatiza que o caminho $C$ é fechado. Nesse caso, supondo $\vec{F}=P\vec{i}+Q\vec{j}$, podemos escrever o teorema da forma
$$
\oint_C  \vec{F} \cdot d\vec{r}=\iint_{D} \left(\frac{\partial Q}{\partial x} - \frac{\partial P}{\partial
y}\right) dA.
$$
\end{teo}


