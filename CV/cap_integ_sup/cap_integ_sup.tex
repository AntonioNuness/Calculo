%Este trabalho está licenciado sob a Licença Creative Commons Atribuição-CompartilhaIgual 3.0 Não Adaptada. Para ver uma cópia desta licença, visite https://creativecommons.org/licenses/by-sa/3.0/ ou envie uma carta para Creative Commons, PO Box 1866, Mountain View, CA 94042, USA.
% !TEX root = ../main.tex
\chapter{Integral de Superfície}
  Neste capítulo, estudamos a integral de superfície\index{integral de superfície} e teoremas relacionados.
\section{Definição de integral de superfície para um campo escalar}

Seja $h: \mathbb{R}^3\to \mathbb{R}$, $h= h(x,y,z)$, um campo escalar definido em todos os pontos de uma superfície regular $S$. Assumimos que $\vec{r}(u,v)=x(u,v)\vec{i}+y(u,v)\vec{j}+z(u,v)\vec{k}$, $(u,v)\in R \subset \mathbb{R}^2$ seja uma parametrização para $S$. A integral de superfície de $h$ sobre $S$ é definida por:
\begin{equation}\label{definicao_int_sup_esc}
\iint_S h d S = \iint_R h(x(u,v),y(u,v),z(u,v)) \|\vec{r}_u\times \vec{r}_v\|d udv, 
\end{equation}
onde $dS$ é o elemento infinitesimal de área sobre a superfície.


\begin{ex}\label{ex_esfera_int_sup}Considere a esfera de raio $a$ 
$$
\vec{r}=a\sin(u)\cos(v)\vec{i}+a\sin(u)\sin(v)\vec{j}+a\cos(u)\vec{k},~~ a>0, ~ 0\leq u< \pi, ~ 0\leq v< 2\pi
$$
e o campo $h(x,y,z)=1$. Vamos calcular a área da esfera dada por $\iint_S h d S $. Começamos com as derivadas parciais de $\vec{r}$, dadas por
$$
\vec{r}_u=a\cos(u)\cos(v)\vec{i}+a\cos(u)\sin(v)\vec{j}-a\sin(u)\vec{k}
$$
e
$$
\vec{r}_v=-a\sin(u)\sin(v)\vec{i}+a\sin(u)\cos(v)\vec{j}.
$$
Então, temos
\begin{eqnarray}
\label{prod_vet_ru_rv}  \vec{r}_u\times\vec{r}_v&=&\left|
 \begin{array}{ccc}
 \vec{i} & \vec{j} & \vec{k} \\
  a\cos(u)\cos(v)&a\cos(u)\sin(v)&-a\sin(u) \\
-a\sin(u)\sin(v)&a\sin(u)\cos(v)&0
 \end{array}
\right|\\
\nonumber &=& a^2\sin^2(u)\cos(v)\vec{i}+ a^2\sin^2(u)\sin(v)\vec{j}\\\nonumber&+&(a^2\cos(u)\sin(u)\cos^2(v)+ a^2\cos(u)\sin(u)\sin^2(v))\vec{k}\\
\nonumber &=& a^2\sin^2(u)\cos(v)\vec{i}+ a^2\sin^2(u)\sin(v)\vec{j}+a^2\cos(u)\sin(u)\vec{k}.
\end{eqnarray}
e
\begin{eqnarray*}
\|  \vec{r}_u\times\vec{r}_v\|&=&a^2\sqrt{ \sin^4(u)(\cos^2(v)+ \sin^2(v))+\cos^2(u)\sin^2(u)}\\
&=&a^2\sqrt{ \sin^4(u)+\cos^2(u)\sin^2(u)}\\
&=&a^2\sin(u)\sqrt{ \sin^2(u)+\cos^2(u)}\\
&=&a^2\sin(u).
\end{eqnarray*}
onde usamos que $0\leq \sin(u)\leq 1$, visto que $0\leq u\leq \pi$. Portanto,
\begin{eqnarray*}
\iint_S h d S &=& \iint_R h(x(u,v),y(u,v),z(u,v)) \|\vec{r}_u\times \vec{r}_v\|d udv \\
&=& \int_0^{2\pi}\int_0^{\pi} a^2\sin(u) d udv \\
&=&2\pi a^2 \left[ -\cos(u) \right]_0^{\pi}\\
&=&4\pi a^2.
\end{eqnarray*}

\end{ex}
\subsection{Superfície definida como função de duas variáveis}
Nessa seção, vamos calcular a versão particular da definição \eqref{definicao_int_sup_esc} quando a superfície é definida como função de duas variáveis, isto é, $z=f(x,y)$ ou $y=f(x,z)$ ou ainda $x=f(y,z)$. Considere o caso onde a superfície $S$ é dado pela função $f:D\subset\mathbb{R}$, $D\subset \mathbb{R}^2$, $z=f(x,y)$ (os outros dois casos são análogos). Uma parametrização para a superfície $S$ é
$$
\vec{r}(x,y)= x\vec{i}+ y\vec{j}+ f(x,y)\vec{k},
$$
Calculamos as derivadas $\vec{r}_x=\vec{i}+ f_x\vec{k},$ e $\vec{r}_y=\vec{j}+ f_y\vec{k}$ e fazemos
$$
  \vec{r}_x\times\vec{r}_y=\left|
 \begin{array}{ccc}
 \vec{i} & \vec{j} & \vec{k} \\
  1 &0 & f_x \\
0 &  1 & f_y
 \end{array}
\right|\\= f_x\vec{i}+ f_y\vec{j}+\vec{k}.
$$
Agora, definimos $G$ tal que a superfície seja reescrita como $G(x,y,z)=0$, isto é, $G(x,y,z)=z-f(x,y)$ e observamos que
\begin{equation}\label{normal_grad_G_0}
   \vec{r}_x\times\vec{r}_y=\vec{ \nabla} G.
\end{equation}
Analogamente, em qualquer dos outros dois casos $y=f(x,z)$ ou $x=f(y,z)$, definimos $G$ tal que a superfície seja reescrita como $G(x,y,z)=0$ e a expressão \eqref{normal_grad_G_0} continua válida, isto é,
\begin{equation*}\label{normal_grad_G_0_2}
   \vec{r}_x\times\vec{r}_z=\vec{ \nabla} G\qquad \text{ou} \qquad    \vec{r}_y\times\vec{r}_z=\vec{ \nabla} G.
\end{equation*}
Portanto, a versão da definição \eqref{definicao_int_sup_esc} para o caso onde a superfície é definida por uma função de duas variáveis $f$ é dada por
\begin{equation}\label{definicao_int_sup_esc_par}
\iint_S h d S=  \iint_R h\| \vec{\nabla}G\| dA,
\end{equation}
onde $R$ é o domínio de $h$ e $h$ deve estar sobre os pontos da superfície.

\begin{ex}Vamos calcular novamente a integral de supefície do exemplo \ref{ex_esfera_int_sup}. Considere a equação da esfera de raio $a$, $x^2+y^2+z^2=a^2$ e o campo $h(x,y,z)=1$. Vamos calcular a área da superfície da esfera dada por $\iint_S h d S $. Para aplicar a expressão \eqref{definicao_int_sup_esc_par}, separamos a esfera em duas superfícies: $S_1$ de equação $z=h_1(x,y)=\sqrt{a^2-x^2-y^2}$ e $S_2$ de equação $z=h_2(x,y)=-\sqrt{a^2-x^2-y^2}$. Assim,
\begin{equation*}
\iint_S h d S=  \iint_{S_1} h d S+\iint_{S_2} h d S.
\end{equation*}
Definimos $G_1=z-\sqrt{a^2-x^2-y^2}$ e $G_2=z+\sqrt{a^2-x^2-y^2}$ e calculamos
$$
\vec{\nabla}G_1=\frac{x}{\sqrt{a^2-x^2-y^2}}\vec{i}+\frac{y}{\sqrt{a^2-x^2-y^2}}\vec{j}+\vec{k}
$$
e 
$$
\vec{\nabla}G_2=-\frac{x}{\sqrt{a^2-x^2-y^2}}\vec{i}-\frac{y}{\sqrt{a^2-x^2-y^2}}\vec{j}+\vec{k}
$$
Logo,
$$
\|\vec{\nabla}G_1\|=\|\vec{\nabla}G_2\|=\sqrt{\frac{x^2+y^2}{a^2-x^2-y^2}+1}=\sqrt{\frac{x^2+y^2+a^2-x^2-y^2}{a^2-x^2-y^2}}=\frac{a}{\sqrt{a^2-x^2-y^2}}.
$$
Como $h(x,y,z)=1$, temos:
\begin{equation*}
\iint_S h d S=  \iint_{D} \frac{a}{\sqrt{a^2-x^2-y^2}}d A+\iint_{D}\frac{a}{\sqrt{a^2-x^2-y^2}}dA=2\iint_{D} \frac{a}{\sqrt{a^2-x^2-y^2}}dA,
\end{equation*}
onde $D$ é o disco de raio $a$ no plano $xy$. Portanto, integrando em coordenadas polares, temos:
\begin{eqnarray*}
\iint_S h d S&=& 2 \iint_{D} \frac{a}{\sqrt{a^2-x^2-y^2}} dydx\\
&=& 2 \int_0^{2\pi}\int_{0}^a \frac{a}{\sqrt{a^2-r^2}} rdrd\theta\\
&=& 2\pi a \int_{0}^a \frac{2r}{\sqrt{a^2-r^2}} dr\\
&=& 2\pi a \left[\frac{-\sqrt{a^2-r^2}}{1/2}\right]_{0}^a\\
&=& 4\pi a^2.
\end{eqnarray*}

\end{ex}

\section{Definição de integral de superfície para um campo vetorial}
Considere $S$ uma superfície orientável e $\vec{r}(u,v)=x(u,v)\vec{i}+y(u,v)\vec{j}+z(u,v)\vec{k}$, $(u,v)\in R \subset \mathbb{R}^2$ uma parametrização regular, sendo $\vec{n}=\frac{\vec{r}_u\times \vec{r}_v}{\|\vec{r}_u\times \vec{r}_v\|}$ o vetor normal à $S$. Seja $\vec{F}:\mathbb{R}^3\to \mathbb{R}^3$, $\vec{F}=\vec{F}(x,y,z)$ um campo vetorial definido em todos em pontos de $S$. Então definimos a integral de superfície do campo vetorial $\vec{F}$ sobre $S$ como:
\begin{equation}\label{definicao_int_sup_vet}
\iint_S \vec{F}\cdot \vec{n} d S= \iint_R \vec{F}(x(u,v),y(u,v),z(u,v))\cdot (\vec{r}_u\times \vec{r}_v) dudv.
\end{equation}
\begin{ex}\label{ex_esfera_int_sup_vet}Seja $S$ a esfera de raio $a$ parametrizada por 
$$
\vec{r}=a\sin(u)\cos(v)\vec{i}+a\sin(u)\sin(v)\vec{j}+a\cos(u)\vec{k},~~ a>0, ~ 0\leq u< \pi, ~ 0\leq v< 2\pi
$$
e o campo $\vec{F}(x,y,z)=x\vec{i}+y\vec{j}+z\vec{k}$. Vamos calcular o fluxo do campo $\vec{F}$ através da superfície $S$ dado por $\Phi=\iint_S \vec{F}\cdot\vec{n} d S $. O produto vetorial
$$
\vec{r}_u\times\vec{r}_v= a^2\sin^2(u)\cos(v)\vec{i}+ a^2\sin^2(u)\sin(v)\vec{j}+a^2\cos(u)\sin(u)\vec{k}
$$
foi calculado na equação \eqref{prod_vet_ru_rv}. Observe que a parametrização define uma orientação para fora da superfície. De fato, basta pegar um ponto, por exemplo, $u=v=\pi/2$ e calcular $\vec{r}=a\vec{j}$ e $\vec{r}_u\times\vec{r}_v=a^2\vec{j}$, ou seja, no ponto $(0,a,0)$, a normal aponta para fora. Como
\begin{eqnarray*}
\vec{F}(x(u,v),y(u,v),z(u,v))&=&x(u,v)\vec{i}+y(u,v)\vec{j}+z(u,v)\vec{k}\\
&=&a\sin(u)\cos(v)\vec{i}+a\sin(u)\sin(v)\vec{j}+a\cos(u)\vec{k},
\end{eqnarray*}
temos
\begin{eqnarray*}
 \vec{F}\cdot (\vec{r}_u\times\vec{r}_v)&=& a^3\sin^3(u)\cos^2(v)+a^3\sin^3(u)\sin^2(v)+a^3\cos^2(u)\sin(u)\\
 &=& a^3\sin^3(u)+a^3\cos^2(u)\sin(u)\\
 &=& a^3\sin(u).
\end{eqnarray*}
Portanto,
\begin{eqnarray*}
 \Phi&=&\iint_S \vec{F}\cdot\vec{n} d S \\
&=& \iint_R \vec{F}\cdot (\vec{r}_u\times\vec{r}_v) d udv \\
&=& \int_0^{2\pi}\int_0^\pi a^3\sin(u) d udv \\
&=& 2\pi a^3\int_0^\pi \sin(u) d u \\
&=&4\pi a^3.
\end{eqnarray*}
\end{ex}
\subsection{Superfície definida como função de duas variáveis}
Nessa seção, vamos calcular a versão particular da definição \eqref{definicao_int_sup_vet} quando a superfície é definida como função de duas variáveis, isto é, $z=f(x,y)$ ou $y=f(x,z)$ ou ainda $x=f(y,z)$. Considere o caso onde a superfície $S$ é dado pela função $f:D\subset\mathbb{R}$, $D\subset \mathbb{R}^2$, $z=f(x,y)$ (os outros dois casos são análogos). Uma parametrização para a superfície $S$ é
$$
\vec{r}(x,y)= x\vec{i}+ y\vec{j}+ f(x,y)\vec{k},
$$
onde o vetor normal a superfície é
$$
\vec{n}=\pm\frac{\vec{r}_x\times \vec{r}_y}{\|\vec{r}_x\times \vec{r}_y\|}.
$$
Aqui, o sinal $\pm$ é escolhido para ajustar a orientação da parametrização à orientação da superfície definida por $f$. Calculamos as derivadas $\vec{r}_x=\vec{i}+ f_x\vec{k},$ e $\vec{r}_y=\vec{j}+ f_y\vec{k}$ e fazemos
$$
  \vec{r}_x\times\vec{r}_y=\left|
 \begin{array}{ccc}
 \vec{i} & \vec{j} & \vec{k} \\
  1 &0 & f_x \\
0 &  1 & f_y
 \end{array}
\right|\\= f_x\vec{i}+ f_y\vec{j}+\vec{k}.
$$
Agora, definimos $G$ tal que a superfície seja reescrita como $G(x,y,z)=0$, isto é, $G(x,y,z)=z-f(x,y)$ e observamos que
\begin{equation}\label{normal_grad_G}
   \vec{r}_x\times\vec{r}_y=\vec{ \nabla} G.
\end{equation}
Analogamente, em qualquer dos outros dois casos $y=f(x,z)$ ou $x=f(y,z)$, definimos $G$ tal que a superfície seja reescrita como $G(x,y,z)=0$ e a expressão \eqref{normal_grad_G} continua válida, isto é,
\begin{equation*}\label{normal_grad_G_2}
   \vec{r}_x\times\vec{r}_z=\vec{ \nabla} G\qquad \text{ou} \qquad    \vec{r}_y\times\vec{r}_z=\vec{ \nabla} G.
\end{equation*}
Portanto, a versão da definição \eqref{definicao_int_sup_vet} para o caso onde a superfície é definida por uma função de duas variáveis $f$ é dada por
\begin{equation}\label{definicao_int_sup_vet_par}
\iint_S \vec{F}\cdot \vec{n} d S= \pm \iint_R \vec{F}\cdot \vec{\nabla}G dA,
\end{equation}
onde $R$ é o domínio de $f$, $\vec{F}$ deve estar sobre os pontos da superfície e o sinal $\pm$ deve ser escolhido para que $\pm \vec{\nabla}G$ esteja no sentido da orientação da superfície.

\begin{ex}Vamos calcular o fluxo do campo $\vec{F}=x^2\vec{i}+3y^2\vec{j}$ através da superfície plana $S$ dada pela equação $x+y+z=1$, limitada ao primeiro octante e orientada no sentido da origem em direção ao primeiro octante. Primeiro, definimos $z=f(x,y)=1-x-y$ e $G=z-1+x+y$ e calculamos $\vec{\nabla}G=\vec{i}+\vec{j}+\vec{k}$. Observamos que a orientação de $\vec{\nabla}G$ está no mesmo sentido que o vetor normal, então escolhemos o sinal positivo na expressão \eqref{definicao_int_sup_vet_par}. Também, observe que o campo é constante na variável $z$, ou seja, $\vec{F}(x,y,z)=\vec{F}(x,y,1-x-y)=x^2\vec{i}+3y^2\vec{j}$. Portanto,
\begin{eqnarray*}
\iint_S \vec{F}\cdot \vec{n} d S&=& + \iint_R \vec{F}\cdot \vec{\nabla}G dA\\
&=&  \int_0^1\int_0^{1-x} x^2+3y^2 dydx\\
&=&  \int_0^1  \left[x^2y+y^3\right]_{y=0}^{y=1-x} dx\\
&=&  \int_0^1  x^2(1-x)+(1-x)^3 dx\\
&=&  \left[  \frac{x^3}{3}-\frac{x^4}{4}-\frac{(1-x)^4}{4} \right]_0^1\\
&=&  \left(  \frac{1}{3}-\frac{1}{4}-0\right)-\left(0-0-\frac{1}{4}\right)=\frac{1}{3}.
\end{eqnarray*}
\end{ex}

\begin{ex}Nesse exemplo vamos calcular o fluxo do campo $\vec{F}=3z^2\vec{i}+6\vec{j}+6xz\vec{k}$ através da superfície $S$ dada pela equação $y=x^2$, $0\leq x\leq 2$, $0\leq z\leq 3$, orientada no sentido em que a componente $\vec{i}$ do vetor normal é sempre positiva. Primeiro, definimos $y=f(x,z)=x^2$ e $G=y-x^2$ e calculamos $\vec{\nabla}G=-2x\vec{i}+\vec{j}$. Observamos que a orientação de $\vec{\nabla}G$ está com componente $\vec{i}$ negativa, então escolhemos o sinal negativo na expressão \eqref{definicao_int_sup_vet_par}. Também, observe que o campo é constante na variável $y$, ou seja, $\vec{F}(x,y,z)=\vec{F}(x,x^2,z)=3z^2\vec{i}+6\vec{j}+6xz\vec{k}$. Portanto,
\begin{eqnarray*}
\iint_S \vec{F}\cdot \vec{n} d S&=& - \iint_R \vec{F}\cdot \vec{\nabla}G dA\\
&=&  -\int_0^2\int_0^{3} -6xz^2+6 dzdx\\
&=&  -\int_0^2\left[ -2xz^3+6z\right]_{z=0}^{z=3} dx\\
&=&  -\int_0^2 -54x+18 dx\\
&=&  -\left[ -27x^2+18x\right]_0^2\\
&=&  -\left( -108+36\right)=72.
\end{eqnarray*}
\end{ex}

\begin{ex}\label{ex_esfera_int_sup_vet_2}Nesse exemplo, vamos recalcular o fluxo do exemplo \ref{ex_esfera_int_sup_vet}. Seja $S$ a esfera de raio $a$ dada por 
$$
x^2+y^2+z^2=a^2,
$$
orientada para fora, e o campo $\vec{F}(x,y,z)=x\vec{i}+y\vec{j}+z\vec{k}$. Vamos calcular o fluxo do campo $\vec{F}$ através da superfície $S$ dado por $\Phi=\iint_S \vec{F}\cdot\vec{n} d S $. Para aplicar a expressão \eqref{definicao_int_sup_vet_par}, separamos a esfera em duas superfícies: $S_1$ de equação $z=h_1(x,y)=\sqrt{a^2-x^2-y^2}$ e $S_2$ de equação $z=h_2(x,y)=-\sqrt{a^2-x^2-y^2}$. Assim,
\begin{equation*}
\iint_S \vec{F}\cdot\vec{n} d S=  \iint_{S_1} \vec{F}_1\cdot\vec{n} d S+\iint_{S_2} \vec{F}_2\cdot\vec{n} d S.
\end{equation*}
Definimos $G_1=z-\sqrt{a^2-x^2-y^2}$ e $G_2=z+\sqrt{a^2-x^2-y^2}$ e calculamos
$$
\vec{\nabla}G_1=\frac{x}{\sqrt{a^2-x^2-y^2}}\vec{i}+\frac{y}{\sqrt{a^2-x^2-y^2}}\vec{j}+\vec{k}
$$
e 
$$
\vec{\nabla}G_2=-\frac{x}{\sqrt{a^2-x^2-y^2}}\vec{i}-\frac{y}{\sqrt{a^2-x^2-y^2}}\vec{j}+\vec{k}.
$$
Observe que $\vec{\nabla}G_1$ está no sentido para fora da superfície e $\vec{\nabla}G_2$ no sentido para dentro. Logo, vamos escolher um sinal positivo e outro negativo na definição \eqref{definicao_int_sup_vet_par}.
Assim, sabendo que o campo na superfície $S_1$ é dado por $\vec{F}_1=x\vec{i}+y\vec{j}+\sqrt{a^2-x^2-y^2}\vec{k}$ e, o campos na supefície $S_2$ é dado por $\vec{F}_2=x\vec{i}+y\vec{j}-\sqrt{a^2-x^2-y^2}\vec{k}$, temos:
\begin{eqnarray*}
\iint_S \vec{F}\cdot\vec{n} d S&=&  \iint_{D} \vec{F}_1\cdot \vec{\nabla}G_1 d A-\iint_{D} \vec{F}_2\cdot\vec{\nabla}G_2 d A\\
&=&  \iint_{D} \frac{x^2}{\sqrt{a^2-x^2-y^2}}+\frac{y^2}{\sqrt{a^2-x^2-y^2}}+\sqrt{a^2-x^2-y^2} d A\\&-&\iint_{D} -\frac{x^2}{\sqrt{a^2-x^2-y^2}}-\frac{y^2}{\sqrt{a^2-x^2-y^2}}-\sqrt{a^2-x^2-y^2} d A\\
&=&  2\iint_{D} \frac{x^2+y^2}{\sqrt{a^2-x^2-y^2}}+\sqrt{a^2-x^2-y^2} d A\\
&=&  2\iint_{D} \frac{a^2}{\sqrt{a^2-x^2-y^2}} d A,
\end{eqnarray*}
onde $D$ é o disco de raio $a$ no plano $z=0$. Portanto,
\begin{eqnarray*}
\Phi&=& \iint_S \vec{F}\cdot\vec{n} d S\\
&=&  2\int_0^{2\pi }\int_{0}^a\frac{a^2}{\sqrt{a^2-r^2}} rdrd\theta\\
&=&  2\pi a^2\left[ -\frac{\sqrt{a^2-r^2}}{1/2} \right]_{0}^a\\
&=&  4\pi a^3. 
\end{eqnarray*}


\end{ex}

\begin{ex}\label{int_sup_ex_1} Considere a superfície $S$
$$z=0,~ {x^2+y^2}\leq 1,$$
orientada no sentido decrescente do eixo $z$, isto é, $\vec{n}=-\vec{k}$. Considere também o campo vetorial dado por $\vec{F}=(2+z^2+x)\vec{k}$. Vamos calcular o valor do fluxo $\Phi=\iint_S\vec{F}\cdot \vec{n} dS$.

Como $z=f(x,y)=0,~ {x^2+y^2}\leq 1$, definimos a função $G(x,y,z) = z$ e calculamos
$$
\vec{\nabla} G=\vec{k}.
$$
Observe que $\vec{\nabla} G$ e $\vec{n}$ estão em sentidos oposto e, portanto, na aplicação da expressão \eqref{definicao_int_sup_vet_par}, vamos escolher o sinal negativo. Aplicamos o campo $\vec{F}=(2+z^2+x)\vec{k}$ sobre a superfície $z=0$ e obtemos $\vec{F}=(2+x)\vec{k}$. Assim,
\begin{eqnarray*}
\Phi&:=&\iint_{S} \vec{F}\cdot \vec{n}dS\\
&=&-\iint_A \vec{F}\cdot \vec{\nabla} G dA\\
&=&-\iint_A \left(2+x\right) dA
\end{eqnarray*}
Resolvemos em coordenadas polares da seguinte forma:
\begin{eqnarray*}
\Phi&=&-\int_0^1\int_0^{2\pi} \left(2+r\cos(\theta)\right) rd\theta dr\\
&=&-\int_0^1\int_0^{2\pi} \left(2r+r^2\cos(\theta)\right) d\theta dr\\
&=&-2\pi\int_0^{1} 2r dr-\int_0^1\left(\left[r^2\sin(\theta)\right]_{\theta=0}^{\theta=2\pi} \right)dr \\
&=&-2\pi.
\end{eqnarray*}
\end{ex}

\begin{ex}\label{int_sup_ex_2} Considere a superfície $S$
 $$z=f(x,y)=1- \sqrt{x^2+y^2},$$
orientada na direção crescente do eixo $z$, isto é, o vetor $\vec{n}$ tem componente na direção $\vec{k}$ sempre positiva. Considere também o campo vetorial dado por
$\vec{F}=(2+z^2+x)\vec{k}$. Vamos calcular o valor do fluxo $\Phi=\iint_S\vec{F}\cdot \vec{n} dS$.

Como $z=f(x,y)=1- \sqrt{x^2+y^2}$, definimos $G(x,y,z) = z-1+ \sqrt{x^2+y^2}$ e calculamos
$$
\vec{\nabla} G=\frac{x}{\sqrt{x^2+y^2}}\vec{i}+\frac{y}{\sqrt{x^2+y^2}}\vec{j}+\vec{k}.
$$
Observe que $\vec{\nabla} G$ e $\vec{n}$ estão no mesmo sentido e, portanto, na aplicação da expressão \eqref{definicao_int_sup_vet_par}, vamos escolher o sinal positivo. Aplicamos o campo $\vec{F}=(2+z^2+x)\vec{k}$ sobre a superfície $z=0$ e obtemos $\vec{F}=(2+(1-\sqrt{x^2+^2})^2+x)\vec{k}$. Assim,


\begin{eqnarray*}
\Phi&:=&\iint_{S} \vec{F}\cdot \vec{n}dS\\
&=&\iint_A \vec{F}\cdot \vec{\nabla} G dA\\
&=&\iint_A \left(2+\left(1- \sqrt{x^2+y^2}\right)^2+x\right) dA
\end{eqnarray*}
Resolvemos em coordenadas polares e obtemos
\begin{eqnarray*}
\Phi&=&\int_0^1\int_0^{2\pi} \left(2+(1- r)^2+r\cos(\theta)\right) rd\theta dr\\
&=&\int_0^1\int_0^{2\pi} \left(3r- 2r^2 + r^3+r^2\cos(\theta)\right) d\theta dr\\
&=&2\pi\int_0^{1}\left(3r- 2r^2 + r^3\right) dr+\int_0^1\left(\left[r^2\sin(\theta)\right]_{\theta=0}^{\theta=2\pi} \right)dr \\
&=&2\pi\left[3\frac{r^2}{2}- 2\frac{r^3}{3} + \frac{r^4}{4}\right]_0^1\\
&=&2\pi\left[\frac{3}{2}- \frac{2}{3} + \frac{1}{4}\right]_0^1=\frac{13\pi}{6}.
\end{eqnarray*}

\end{ex}


\subsection*{Exercícios resolvidos}
\begin{exeresol} Seja $S$ a porção da superfície $z=1-x^2-y^2$, acima do plano $xy$, orientada com componente $\vec{k}$ do vetor normal positiva. Encontre o fluxo $\Phi$ através de $S$ para o campo $\vec{F}=x\vec{i}+y\vec{j}+z\vec{k}$.
\end{exeresol}
\begin{resol}
Definimos $z=f(x,y)=1-x^2-y^2$, $G=z-1+x^2+y^2$ e calculamos $\vec{\nabla}G=2x\vec{i}+2y\vec{j}+\vec{k}$. Também, como a componente $\vec{k}$ desse gradiente é positiva, escolheremos a sinal positivo na expressão \eqref{definicao_int_sup_vet_par}. Colocamos o campo sobre a superfície, isto é, $\vec{F}=x\vec{i}+y\vec{j}+z\vec{k}=x\vec{i}+y\vec{j}+(1-x^2-y^2)\vec{k}$ e obtemos
\begin{eqnarray*}
\Phi&:=&\iint_{S} \vec{F}\cdot \vec{n}dS\\
&=&+\iint_D \vec{F}\cdot \vec{\nabla} G dA,
\end{eqnarray*}
onde $D$ é o disco de raio unitário no plano $z=0$. Portanto,
\begin{eqnarray*}
\Phi&=&\iint_D (x\vec{i}+y\vec{j}+(1-x^2-y^2)\vec{k})\cdot (2x\vec{i}+2y\vec{j}+\vec{k}) dA\\
&=&\iint_D (2x^2+2y^2+(1-x^2-y^2))dA\\
&=&\iint_D (x^2+y^2+1)dA\\
&=&\int_0^{2\pi}\int_0^1 (r^2+1)rdrd\theta\\
&=&2\pi \left[ \frac{r^4}{4}+\frac{r^2}{2}\right]_0^1=\frac{6\pi}{4}.
\end{eqnarray*}


\end{resol}

\begin{exeresol} Dados o campo vetorial $\vec{F}=\vec{i}+\vec{j}+\vec{k}$ e a porção da superfície $S$ de equação $z=\sqrt{x^2+y^2}$, situada abaixo do plano $z=1$, orientada com componente $\vec{k}$ do vetor normal positiva, calcule o fluxo $\Phi=\iint_{S} \vec{F}\cdot \vec{n}dS$.
\end{exeresol}
\begin{resol}
Definimos $z=f(x,y)=\sqrt{x^2+y^2}$, $G=z-\sqrt{x^2+y^2}$ e calculamos $\vec{\nabla}G=-\frac{x}{\sqrt{x^2+y^2}}\vec{i}-\frac{y}{\sqrt{x^2+y^2}}\vec{j}+\vec{k}$. Também, como a componente $\vec{k}$ desse gradiente é positiva, escolheremos a sinal positivo na expressão \eqref{definicao_int_sup_vet_par}. O campo é constante na variável $z$, isto é, $\vec{F}(x,y,z)=\vec{F}(x,y,f(x,y))=\vec{i}+\vec{j}+\vec{k}$. Então, temos:
\begin{eqnarray*}
\Phi&:=&\iint_{S} \vec{F}\cdot \vec{n}dS\\
&=&+\iint_D \vec{F}\cdot \vec{\nabla} G dA,
\end{eqnarray*}
onde $D$ é o disco de raio unitário no plano $z=0$. Portanto,
\end{resol}
\begin{eqnarray*}
\Phi&=&\iint_D (\vec{i}+\vec{j}+\vec{k})\cdot \left(-\frac{x}{\sqrt{x^2+y^2}}\vec{i}-\frac{y}{\sqrt{x^2+y^2}}\vec{j}+\vec{k}\right) dA\\
&=&\iint_D \left(-\frac{x}{\sqrt{x^2+y^2}}-\frac{y}{\sqrt{x^2+y^2}}+1\right) dA\\
&=&\int_0^{2\pi}\int_0^1 \left(-\frac{r\cos(\theta)+r\sin(\theta)}{r}+1\right) rdrd\theta\\
&=&\int_0^{2\pi}\int_0^1 r\left(-\cos(\theta)-\sin(\theta)+1\right) drd\theta\\
&=&\int_0^{2\pi}\left[\left(-\cos(\theta)-\sin(\theta)+1\right)\frac{r^2}{2}\right]_{r=0}^{r=1} d\theta\\
&=&\frac{1}{2}\left[-\sin(\theta)+\cos(\theta)+\theta\right]_0^{2\pi}=\pi.
\end{eqnarray*}




\begin{exeresol} Seja $S$ a superfície do cubo dado pelos planos $x=\pm 1$, $y=\pm 1$ e $z=\pm 1$, orientada para fora. Calcule o fluxo através de $S$ para os seguintes campos vetoriais:
\begin{itemize}
 \item[a)]$\vec{F}=x\vec{i}$
 \item[b)]$\vec{F}=x\vec{i}+y\vec{j}+z\vec{k}$
 \item[c)]$\vec{F}=x^2\vec{i}+y^2\vec{j}+z^2\vec{k}$
 \end{itemize}
\end{exeresol}
\begin{resol}
Em cada um dos itens, temos que calcular seis integrais de superfície em cada face  do cubo, a saber, $S_1:x=1$, $S_2:x=-1$, $S_3:y=1$, $S_4:y=-1$, $S_5:z=1$ e $S_6:z=-1$. 
\begin{itemize}
 \item[a)] Começamos com a face $S_1$, definimos $x=f_1(y,z)=1$, $G_1=x-1$ e calculamos $\vec{\nabla}G_1=\vec{i}$. Observe que $\vec{\nabla}G_1$ está orientado no sentido de $\vec{n}=\vec{i}$ e $\vec{F}=x\vec{i}=\vec{i}$. Calculamos
 $$
 \iint_{S_1}\vec{F}\cdot \vec{n}dS= \iint_{S_1}dA=4,
 $$
 onde $4$ é a área do quadrado de lado $2$. Analogamente,
$$
 \iint_{S_2}\vec{F}\cdot \vec{n}dS= \iint_{S_2}(-\vec{i})\cdot (-\vec{i})dS=\iint_{S_2}dA=4.
 $$
 As demais integrais são nulas, visto que o fluxo só tem componente na direção $\vec{i}$. Logo,
 $$
 \iint_{S}\vec{F}\cdot \vec{n}dS= 8.
 $$
  \item[b)] Começamos com a face $S_1$, definimos $x=f_1(y,z)=1$, $G_1=x-1$ e calculamos $\vec{\nabla}G_1=\vec{i}$. Observe que $\vec{\nabla}G_1$ está orientado no sentido de $\vec{n}=\vec{i}$ e $\vec{F}=x\vec{i}+y\vec{i}+z\vec{k}=\vec{i}+y\vec{i}+z\vec{k}$. Calculamos
 $$
 \iint_{S_1}\vec{F}\cdot \vec{n}dS= \iint_{S_1}dA=4,
 $$
 onde $4$ é a área do quadrado de lado $2$. Analogamente,
$$
 \iint_{S_2}\vec{F}\cdot \vec{n}dS= \iint_{S_2}(-\vec{i}+y\vec{i}+z\vec{k})\cdot (-\vec{i})dS=\iint_{S_2}dA=4.
 $$
Pela simetria do problema, vemos que as demais integrais são todas iguais a $4$. Logo,
 $$
 \iint_{S}\vec{F}\cdot \vec{n}dS= 24.
 $$
  \item[c)] Começamos com a face $S_1$, definimos $x=f_1(y,z)=1$, $G_1=x-1$ e calculamos $\vec{\nabla}G_1=\vec{i}$. Observe que $\vec{\nabla}G_1$ está orientado no sentido de $\vec{n}=\vec{i}$ e $\vec{F}=x^2\vec{i}+y^2\vec{i}+z^2\vec{k}=\vec{i}+y^2\vec{i}+z^2\vec{k}$. Calculamos
 $$
 \iint_{S_1}\vec{F}\cdot \vec{n}dS= \iint_{S_1}dA=4,
 $$
 onde $4$ é a área do quadrado de lado $2$. Analogamente, como em $S_2:\ \!x=-1$ o campo assume a forma $\vec{F}=x^2\vec{i}+y^2\vec{i}+z^2\vec{k}=\vec{i}+y^2\vec{i}+z^2\vec{k}$, temos
$$
 \iint_{S_2}\vec{F}\cdot \vec{n}dS= \iint_{S_2}(\vec{i}+y^2\vec{i}+z^2\vec{k})\cdot (-\vec{i})dS=\iint_{S_2}dA=4.
 $$
 Observe que
 $$
  \iint_{S_1}\vec{F}\cdot \vec{n}dS+ \iint_{S_2}\vec{F}\cdot \vec{n}dS=0
 $$
Pela simetria do problemas, teremos de forma análoga
 $$
  \iint_{S_3}\vec{F}\cdot \vec{n}dS+ \iint_{S_4}\vec{F}\cdot \vec{n}dS=0
 $$
e
 $$
  \iint_{S_5}\vec{F}\cdot \vec{n}dS+ \iint_{S_6}\vec{F}\cdot \vec{n}dS=0.
 $$
 Logo,
 $$
 \iint_{S}\vec{F}\cdot \vec{n}dS= 0.
 $$
 
 
 \end{itemize}

\end{resol}

\begin{exeresol} Calcule o fluxo de $\vec{F}=z\vec{k}$ através da esfera $x^2+y^2+z^2= a^2$, orientada para fora. \end{exeresol}
\begin{resol}
Começamos separando a esfera nos seguintes dois hemisférios: $S_1:\ \!z=f_1(x,y)=\sqrt{ a^2-x^2-y^2}$ e $S_2:\ \!z=f_2(x,y)=-\sqrt{ a^2-x^2-y^2}$. Definimos $G_1=z-\sqrt{ a^2-x^2-y^2}$ e $G_2=z+\sqrt{ a^2-x^2-y^2}$ e calculamos 
$$
\vec{\nabla}G_1=\frac{x}{\sqrt{a^2-x^2-y^2}}\vec{i}+\frac{y}{\sqrt{a^2-x^2-y^2}}\vec{j}+\vec{k}
$$
e 
$$
\vec{\nabla}G_2=-\frac{x}{\sqrt{a^2-x^2-y^2}}\vec{i}-\frac{y}{\sqrt{a^2-x^2-y^2}}\vec{j}+\vec{k}.
$$
Observe que $\vec{\nabla}G_1$ está orientada para fora e $\vec{\nabla}G_2$ está orientada para dentro. Logo, escolheremos para as integrais sobre $S_1$ e $S_2$ os sinais positivo e negativo, respectivamente, na expressão \eqref{definicao_int_sup_vet_par}. O campo sobre cada umas das superfícies tem às formas: $\vec{F}_1=\sqrt{ a^2-x^2-y^2}\vec{k}$ e $\vec{F}_2=-\sqrt{ a^2-x^2-y^2}\vec{k}$. Assim,
\begin{eqnarray*}
 \Phi&=&\iint_S\vec{F}\cdot \vec{n}dS\\
 &=&\iint_{S_1}\vec{F}_1\cdot \vec{n}dS+\iint_{S_2}\vec{F}_2\cdot \vec{n}dS\\
&=&\iint_{D}\vec{F}_1\cdot \vec{\nabla}G_1 dA-\iint_{D}\vec{F}_2\cdot \vec{\nabla}G_2 dA,
 \end{eqnarray*}
onde $D$ é o disco de raio $a$ no plano $z=0$. Portanto,
\begin{eqnarray*}
 \Phi&=&\iint_{D}\sqrt{ a^2-x^2-y^2} dA-\iint_{D}(-\sqrt{ a^2-x^2-y^2}) dA\\
 &=&2\iint_{D}\sqrt{ a^2-x^2-y^2} dA\\
 &=&2\pi \left[- \frac{(a^2-r^2)^{3/2}}{3/2}\right]_{0}^a=\frac{4\pi a^3}{3}.\\
 \end{eqnarray*}
\end{resol}

\subsection*{Exercícios}
\begin{exer}
 Calcule o fluxo $\Phi$ do campo vetorial $\vec{F}$ através da superfície $S$.
 \begin{itemize}
  \item[a)] $\vec{F}=x\vec{i}+y\vec{j}+2z\vec{k}$ e $S:$ $z=1-x^2-y^2$ acima do plano $xy$ e orientada com componente $\vec{k}$ da normal positiva.
  \item[b)] $\vec{F}=(x+y)\vec{i}+(y+z)\vec{j}+(x+z)\vec{k}$ e $S$ é a porção do plano $x+y+z=1$ no primeiro octante e orientada da origem para o primeiro octante.
  \item[c)] $\vec{F}=z^2\vec{k}$ e $S$ o hemisfério $z=\sqrt{1-x^2-y^2}$ orientada com componente $\vec{k}$ da normal positiva.
  \item[d)] $\vec{F}=x\vec{i}+y\vec{j}+2z\vec{k}$ e $S$ é a porção do cone $z^2=x^2+y^2$ entre os plano $z=1$ e $z=2$, orientada com componente $\vec{k}$ da normal negativa.
 \end{itemize}
\end{exer}
\begin{resp}
 \begin{itemize}
  \item[a)] $2\pi$.
  \item[b)] $1$.
  \item[c)] $\frac{\pi}{2}$.
  \item[d)] $-\frac{14\pi}{3}$. 
 \end{itemize}
\end{resp}




\section{O Teorema da Divergência de Gauss}
\begin{teo}
Seja $V$ o volume de um sólido cuja superfície $S$ é orientada para fora. Seja o campo vetorial $\vec{F}$ dado por
$$
\vec{F}=f(x,y,z)\vec{j}+g(x,y,z)\vec{i}+h(x,y,z)\vec{k},
$$
onde as funções $f$, $g$ e $h$ possuem todas as derivadas parciais de primeira ordem contínuas em algum conjunto aberto contendo $V$. Então:
$$
\oiint_S \vec{F}\cdot \vec{n} dS=\iiint_V \vec{\nabla}\cdot \vec{F} dV.
$$
\end{teo}

\begin{ex}\label{int_sup_ex_div_1} Considere a superfície fechada orientada para fora composta superiormente pela superfície de rotação descrita como
$$z=f(x,y)=1- \sqrt{x^2+y^2}$$ e inferiormente por
$$z=0,~ {x^2+y^2}\leq 1.$$
Seja o campo vetorial dado por
$\vec{F}=(2+z^2+x)\vec{k}$. Vamos calcular o valor do fluxo $\iint_S\vec{F}\cdot \vec{n} dS$ via teorema da divergência.

Temos $\vec{\nabla}\cdot \vec{F}=2z$. Assim,
\begin{eqnarray*}
\Phi&=&\iint_S \vec{F}\cdot \vec{n}dS\\
&=&\iiint_V \vec{\nabla}\cdot \vec{F} dV\\
&=&\int_0^1 \int_0^{2\pi}  \int_0^{1-r} 2z rdzd\theta dr\\
&=&\int_0^1 \int_0^{2\pi}   (1-r)^2 rd\theta dr\\
&=&2\pi \int_0^1    (1-r)^2 r dr\\
&=&2\pi \int_0^1    \left(r-2r^2+r^3\right) dr\\
&=&2\pi  \left[\frac{r^2}{2}-2\frac{r^3}{3}+\frac{r^4}{4}\right]_{0}^1\\
&=&2\pi  \left(\frac{1}{2}-\frac{2}{3}+\frac{1}{4}\right)=\frac{\pi}{6}.
\end{eqnarray*}

Observe que as superfícies dos exemplos \ref{int_sup_ex_1} e \ref{int_sup_ex_1} foram a superfície fechada do exemplo \ref{int_sup_ex_div_1}. De fato, o fluxo calculado naqueles dois exemplos via parametrização direta foram $-2\pi$ e $\frac{13\pi}{6}$, cuja soma é $\frac{\pi}{6}$, o mesmo valor calculado pelo Teorema da Divergência.
\end{ex}


\subsection*{Exercícios resolvidos}

\begin{exeresol} Seja $S$ a superfície do cubo dado pelos planos $x=\pm 1$, $y=\pm 1$ e $z=\pm 1$, orientada para fora. Calcule o fluxo através de $S$ para os seguintes campos vetoriais:
\begin{itemize}
 \item[a)]$\vec{F}=x\vec{i}$
 \item[b)]$\vec{F}=x\vec{i}+y\vec{j}+z\vec{k}$
 \item[c)]$\vec{F}=x^2\vec{i}+y^2\vec{j}+z^2\vec{k}$
 \end{itemize}
\end{exeresol}
\begin{resol}
Vamos utilizar o teorema da divergência de Gauss
$$
\oiint_S \vec{F}\cdot \vec{n} dS=\iiint_V \vec{\nabla}\cdot \vec{F} dV.
$$
\begin{itemize}
 \item[a)] Como $\nabla\cdot \vec{F}=1$, temos: 
$$
\oiint_S \vec{F}\cdot \vec{n} dS=\iiint_V 1 dV=8,
$$
 onde $8$ é o volume do cubo de lado $2$.
 \item[b)] Como $\nabla\cdot \vec{F}=3$, temos: 
$$
\oiint_S \vec{F}\cdot \vec{n} dS=\iiint_V 3 dV=24.
$$
 \item[c)] Como $\nabla\cdot \vec{F}=2x+2y+2z$, temos: 
\begin{eqnarray*}
\oiint_S \vec{F}\cdot \vec{n} dS &=& \iiint_V (2x+2y+2z) dV\\
&=&\int_{-1}^1\int_{-1}^1\int_{-1}^1(2x+2y+2z)dxdydz = 0
\end{eqnarray*}

 \end{itemize}


\end{resol}

\begin{exeresol} Calcule o fluxo de $\vec{F}=z\vec{k}$ através da esfera $x^2+y^2+z^2= a^2$, orientada para fora. \end{exeresol}
\begin{resol}
Como $\nabla\cdot \vec{F}=1$, temos: 
$$
\oiint_S \vec{F}\cdot \vec{n} dS=\iiint_V 1 dV=\frac{4\pi a^3}{3},
$$
onde o valor $\frac{4\pi a^3}{3}$ é o voluma de uma esfera de raio $a$.
\end{resol}

\begin{exeresol} Use o teorema da divergência para calcular o fluxo do campo $\vec{F}=x^3\vec{i}+y^3\vec{j}+z^2\vec{k}$ através da superfície fechada e orientada para fora composta pelos planos $z=0$ e $z=2$ e a superfície $x^2+y^2= 9$.
\end{exeresol}
\begin{resol}
Como $\nabla\cdot \vec{F}=3x^2+3y^2+2z$, temos: 
$$
\oiint_S \vec{F}\cdot \vec{n} dS=\iiint_V 3x^2+3y^2+2z dV.
$$
onde $V$ é a região cilíndrica do enunciado. Vamos integrar em coordenadas cilíndricas:
\begin{eqnarray*}
\oiint_S \vec{F}\cdot \vec{n} dS&=&\int_0^{2\pi} \int_0^3 \int_0^2 3 (r^2+2z )r dzdrd\theta\\
&=&2\pi \int_0^3 \left[ 3 r^3z+z^2r \right]_{z=0}^{z=2}dr\\
&=&2\pi  \left[ \frac{6r^4}{4} +2r^2 \right]_{0}^{3}=279\pi. 
\end{eqnarray*}
\end{resol}

\begin{exeresol} Use o teorema da divergência para calcular o fluxo do campo $\vec{F}=x^3\vec{i}+y^3\vec{j}+z^3\vec{k}$ através da superfície fechada e orientada para fora composta pelo plano $z=0$ e o hemisfério  $z=\sqrt{a^2-x^2-y^2}$.
\end{exeresol}
\begin{resol}
Como $\nabla\cdot \vec{F}=3x^2+3y^2+3z^2$, temos: 
$$
\oiint_S \vec{F}\cdot \vec{n} dS=\iiint_V 3x^2+3y^2+3z^2 dV.
$$
onde $V$ é o hemisfério do enunciado. Vamos integrar em coordenadas esféricas:
\begin{eqnarray*}
\oiint_S \vec{F}\cdot \vec{n} dS&=&\int_0^{2\pi} \int_0^{\pi/2} \int_0^a 3\rho^2 \rho^2\sin(\phi)d\rho d\phi d\theta\\
&=&2\pi \int_0^{\pi/2} \int_0^a 3\rho^4\sin(\phi)d\rho d\phi \\
&=&2\pi \int_0^{\pi/2} \left[ \frac{3\rho^5}{5}\sin(\phi)\right]_{\rho=0}^{\rho=a} d\phi \\
&=&\frac{6\pi a^5}{5} \left[ -\cos(\phi)\right]_0^{\pi/2}=\frac{6\pi a^5}{5}. 
\end{eqnarray*}
\end{resol}


\begin{exeresol} Seja $\Phi$ o fluxo do campo
   $$\vec{F}=z^2\vec{k}$$
   através da superfície que envolve a região limitada inferiormente pelo cone
   $$\sqrt{x^2+y^2}=z$$
   e superiormente pelo plano $z=1$ orientada para fora. 
   Encontre  o fluxo $\Phi$ via parametrização direta da superfície (sem usar o Teorema da Divergência) e, depois,  usando o Teorema da Divergência.
\end{exeresol}
\begin{resol}
   Para calcular via parametrização direta, escrevemos o fluxo $\Phi$ como a soma da contribuição da porção de cone ($\Phi_b$) com a do disco unitário no plano $z=1$ ($\Phi_t$).
$$\Phi_t = \iint_{S_t}\vec{F}\cdot\vec{k}dS =  \iint_{S_t}dS=\pi $$
A porção de cone está numa superfície de nível da seguinte função escalar:
$$G(x,y,z)=z-\sqrt{x^2+y^2} $$
Usaremos o teorema de mudança de variáveis:
$$\Phi_b = \iint_{S_t}\vec{F}\cdot\vec{k}dS =  \pm \iint_{A}\vec{F}\cdot \vec{\nabla}GdA $$
Onde o sinal é determinado conforme a orientação da superfície. Neste caso, escolhemos negativo porque a componente $z$ do gradiente é positiva e a componente $z$ da normal é negativa.
$$\vec{F}\cdot \vec{\nabla}G=z^2$$
\begin{eqnarray*}
 \Phi_b &=&\pm \int_0^{2\pi}\int_0^1(z^2)\rho d\rho d\theta\\
 &=&-\int_0^{2\pi}\int_0^1\rho^3 d\rho d\theta=-\frac{\pi}{2}
\end{eqnarray*}
$$\Phi=\Phi_t+\Phi_b=\frac{\pi}{2}$$

Para calcular via Teorema da Dirvergência, calculamos a seguinte integral de volume:
\begin{eqnarray*}
   \Phi&=&\int_V \vec{\nabla}\cdot \vec{F}dV= \int_0^{2\pi}\int_0^1\int_\rho^1 (2z)\rho dz d\rho d\theta\\
   &=& \int_0^{2\pi}\int_0^1(1-\rho^2)\rho d\rho d\theta\\
   &=& {2\pi}\int_0^1(\rho-\rho^3) d\rho \\
   &=&{2\pi}\left(\frac{1}{2}-\frac{1}{4}\right)=\frac{\pi}{2}
  \end{eqnarray*}
  
\end{resol}
\subsection*{Exercícios}
\begin{exer}
 Seja $\vec{F}=r^k\vec{r}$ onde $k$ é uma constante. Seja $S$ um esfera de raio $a$ centrada na origem e orientada para fora por $\vec{n}=\frac{\vec{r}}{r}=\frac{\vec{r}}{a}$. 
 \begin{itemize}
  \item[a)] Calcule $\oiint_S\vec{F}\cdot \vec{n}dS$.
  \item[b)] Para qual valor de $k$ o valor da integral de superfície fica independente do raio $a$?
 \end{itemize}
\end{exer}
\begin{resp}
 \begin{itemize}
  \item[a)] $4\pi a^{k+3}$.
  \item[b)] $k=-3$ (campo inverso do quadrado).
 \end{itemize} 
\end{resp}

\begin{exer}
 Use o teorema da divergência para calcular o fluxo $\Phi$ do campo $\vec{F}$ através da superfície $S$ fechada e orientada para fora.
 \begin{itemize}
  \item[a)] $\vec{F}=4x\vec{i}-3y\vec{j}+7z\vec{k}$ e $S$ é o cubo limitado pelos planos coordenados e pelos planos $x=1$, $y=1$ e $z=1$.
  \item[b)] $\vec{F}=2x\vec{i}+2y\vec{j}+2z\vec{k}$ e $S$ a esfera $x^2+y^2+z^2=9$.
  \item[c)] $\vec{F}=(x-z)\vec{i}+(y-x)\vec{j}+(z-y)\vec{k}$ e $S$ é a superfície do sólido limitado por $x^2+y^2=a^2$, $z=0$ e $z=1$.
  \item[d)] $\vec{F}=x^3\vec{i}+y^3\vec{j}+z^3\vec{k}$ e $S$ é a superfície do sólido limitado por $x^2+y^2=4$, $z=0$ e $z=3$.
  \item[e)] $\vec{F}=(x^2+y)\vec{i}+xy\vec{j}-(2xz+y)\vec{k}$ e $S$ é a superfície do tetraedro no 1º octante, limitado por $x+y+z=1$ e os planos coordenados.
  \item[f)] $\vec{F}=x^2\vec{i}+y^2\vec{j}+z^2\vec{k}$ e $S$ a superfície do sólido limitado por $z=\sqrt{x^2+y^2}$ e $z=1$.
 \end{itemize}
\end{exer}
\begin{resp}
\begin{itemize}
 \item[a)] $8$.
 \item[b)] $216\pi$. 
 \item[c)] $3\pi a^2$. 
 \item[d)] $180\pi$.
 \item[e)] $\frac{1}{24}$. 
 \item[f)] $\frac{\pi}{2}$. 
\end{itemize}
\end{resp}



\section{O Teorema de Stokes}

\begin{teo}
Seja $S$ uma superfície orientável, suave por partes, limitada por uma curva $C$, fechada, suave por partes e positivamente orientada com respeito a $S$. Seja o campo vetorial $\vec{F}$ dado por
$$
\vec{F}=f(x,y,z)\vec{j}+g(x,y,z)\vec{i}+h(x,y,z)\vec{k},
$$
onde as funções $f$, $g$ e $h$ possuem todas as derivadas parciais de primeira ordem contínuas em algum conjunto aberto contendo $S$. Então:
$$
\oint_C \vec{F}\cdot d \vec{r}=\iint_S \vec{\nabla}\times \vec{F}\cdot \vec{n} dS.
$$
\end{teo}

\subsection{Exercícios resolvidos}
\begin{exeresol}
   Calcule o trabalho realizado pelo campo de forças $\vec{F}=-\cos^2(x)y\vec{i}+(z^2+y^2)\vec{j}$ ao longo do retângulo cujos vértices são $(0,0,0)$, $(\pi,0,0)$,  $(\pi,2,0)$ e $(0,2,0)$ no sentido anti-horário.
\end{exeresol}
\begin{resol}
\begin{eqnarray*}
   W &=& \oint_C \vec{F}\cdot\vec{dr}=\int_S(\vec{\nabla}\times\vec{F})\cdot \vec{N}dS\\
   &=&\int_0^\pi\int_0^2(\vec{\nabla}\times\vec{F})\cdot \vec{k} dy dx 
   \end{eqnarray*}
   Agora, precisamos calcular $\vec{\nabla}\times\vec{F}\cdot\vec{k}$:
   $$\vec{\nabla}\times\vec{F}\cdot\vec{k}=\left|\begin{array}{ccc}
   \vec{i}&\vec{j}&\vec{k}\\
   \frac{\partial}{\partial x}&\frac{\partial}{\partial y}&\frac{\partial}{\partial z}\\
   -\cos^2(x)y&z^2+y^2&0
   \end{array}
   \right|\cdot \vec{k}=\cos^2(x)
   $$
   E finalmente, temos:
   \begin{eqnarray*}
   W&=&\int_0^\pi\int_0^2\cos^2(x)dydx\\
   &=&2\int_0^\pi\cos^2(x)dx=\int_0^\pi(1+\cos(2\phi))dx\\
   &=&\pi
   \end{eqnarray*}
   Onde foi usada a seguinte identidade trigonométrica:
   $$\cos^2\phi=\frac{1+\cos(2\phi)}{2}.$$
\end{resol}
\begin{exeresol}Use o teorema de Stokes para calcular o trabalho realizado pelo campo de força dado por:
   $$\vec{F}=yx^2\cos(z)\vec{i}-z\vec{j}+\sin(xz)\vec{k}$$ 
   ao deslocar uma partícula ao longo da circunferência de raio 1, centrada na origem sobre o plano $xy$ orientada no sentido horário.
\end{exeresol}
   \begin{resol}
      Como a curva está sobre o plano $xy$, o vetor normal unitário deve ser $\vec{k}$ ou $-\vec{k}$. Pela orientação da curva e a regra da mão direita, identificamos que $\vec{n}=-\vec{k}$.
      
      Agora calculamos:
      $$W=\oint_C \vec{F}\cdot \vec{dr} = -\iint_S \vec{\nabla}\times\vec{F}\cdot \vec{k} dA$$
      
      \begin{eqnarray*}
       \vec{k}\cdot\vec{\nabla}\times\vec{F}&=&\frac{\partial F_2}{\partial x}-\frac{\partial F_1}{\partial y}\\
       &=&0-x^2\cos(z)=-x^2,~~z=0
      \end{eqnarray*}
      
      Parametrizamos em coordenadas polares:
      $$x=\rho \cos(\phi),~~ y=\rho \sin(\phi)$$
      \begin{eqnarray*}
      W&=&-\iint_S \vec{\nabla}\times\vec{F}\cdot \vec{k} dA\\
      &=& \int_0^{2\pi}\int_0^1 x^2 \rho d\rho d\phi\\
      &=&\int_0^{2\pi}\int_0^1  \rho^3\cos^2\phi d\rho d\phi\\
      &=&\left(\int_0^1 \rho^3d\rho\right)\left(\int_0^{2\pi}\cos^2 \phi d\phi\right)\\
      &=&\frac{\pi}{4}
      \end{eqnarray*}
      Onde se usou a identidade:
      \begin{eqnarray*}
         \int_0^{2\pi}\cos^2 \phi d\phi
         &=& \int_0^{2\pi}\left(\frac{1+\cos(2\phi)}{2}\right)  d\phi = {\pi}
         \end{eqnarray*}


   \end{resol}

\begin{exeresol}
 Dado $\vec{F}=4y\vec{i}+x\vec{j}+2z\vec{k}$, encontre a maneira mais simples de calcular o fluxo do rotacional $\vec{\nabla}\times\vec{F}$ através do hemisfério $z=\sqrt{a^2-x^2+y^2}$, $a> 0$, orientado com componente $\vec{k}$ da normal positiva.
\end{exeresol}
\begin{resol}
   A ideia aqui é aplicar o teorema de Stokes duas vezes para concluir que o fluxo atráves do hemisfério dado é igual ao fluxo através do disco no plano $z=0$ com vetor normal unitário $\vec{\eta}=\vec{k}$. Assim o fluxo é dado por:
   $$\Phi=\iint_D \vec{\nabla}\vec{F}\cdot{k}dS $$
   Calculamos o integrando:
 \begin{eqnarray*}
 \vec{k}\cdot\vec{\nabla}\vec{F} &=& \frac{\partial F_2}{\partial x} - \frac{\partial F_1}{\partial y} \\
 &=& 1-4=-3
 \end{eqnarray*}
 Portando o fluxo é $-3$ multiplicado pela área do disco:
 $$\Phi=-3\pi a^2. $$
\end{resol}
\begin{exeresol}
Use o teorema de Stokes para calcular o trabalho realizado pelo campo de forças $\vec{F}=x^2\vec{i}+4xy^3\vec{j}+y^2x\vec{k}$ ao longo da curva $C$ que limita o retângulo no plano $z=y$ de vértices $P_0(0,0,0)$, $P_1(1,0,0)$, $P_2(1,3,3)$ e $P_3(0,3,3)$, no sentido $P_0\to P_1\to P_2\to P_3\to P_0$.
\end{exeresol}
\begin{resol}
  Sabemos que o retângulo está no plano $z=y$, pelo que a função $G(x,y,z)$ pode ser escolhida como:
  $$G(x,y,z) = z-y \Longrightarrow \vec{\nabla}G(x,y,z) = \vec{k}-\vec{j}$$

 Agora calculamos o rotacional é dado por:
 $$\vec{\nabla}\times \vec{F} = 3xy\vec{i} -y^2\vec{j}+4y^3\vec{k}.$$
 
 Temos, então:
 \begin{eqnarray*}
   W &=& \iint_S\vec{\nabla}\times \vec{F}\cdot \vec{\eta}dS = \iint_A\vec{\nabla}\times \vec{F}\cdot \vec{\nabla}GdA\\
   &=&\int_0^3\int_0^1\left(y^2+4y^3\right)dx dy\\
   &=&\int_0^3\left(y^2+4y^3\right)dy = 9+ 81=90.
\end{eqnarray*}

\end{resol}


\subsection*{Exercícios}
\begin{exer}
 Calcule a circulação do campo de velocidade $\vec{v}=3z\vec{i}+4x\vec{j}+2y\vec{k}$ ao longo da curva $C$ que limita o parabolóide $z=4-x^2-y^2$, $z\geq 0$. A curva $C$ está no plano $xy$ orientada no sentido anti-horário.
 \end{exer}
\begin{resp}
 $16\pi$
\end{resp}
\begin{exer}Mostre que se $\vec{F}=f(x,y)\vec{i}+g(x,y)\vec{j}$, então
$$
\oint_C  \vec{F} \cdot d\vec{r}=\iint_{D} \left(\frac{\partial Q}{\partial x} - \frac{\partial P}{\partial
y}\right) dA,
$$
isto é, o teorema de Stokes em duas dimensões recai no teorema de Green.
\end{exer}
\begin{exer}Use o teorema de Stokes para mostrar que:
\begin{itemize}
 \item[a)] se $\vec{r}=x\vec{i}+y\vec{j}+z\vec{k}$, então $\oint_C\vec{r}\cdot d\vec{r}=0$.
 \item[b)] $\oint_C\vec{\nabla}\phi\cdot d\vec{r}=0$. 
\end{itemize}
\end{exer}
\begin{exer} Considere a linha poligonal fechada $C$ no plano $xy$ formada pelos pontos $P_0(1,0,1)$, $P_1(3,0,1)$ e $P_2(2,1,1)$, no sentido $P_0\to P_1\to P_2\to P_0$, e o campo $\vec{F}=-(z^3+1)y\vec{i}+(z^2+1)x\vec{j}+xye^z\vec{k}$. Use o teorema de Stokes para calcular a integral de linha $\oint_C \vec{F}\cdot d\vec{r}$.
\end{exer}
\begin{resp}4
\end{resp}
