\chapter{Integral de Superfície}
  Neste capítulo, estudamos a integral de superfície\index{integral de superfície} e teoremas relacionados.
\section{Definição de integral de superfície para um campo escalar}

Seja $h: \mathbb{R}^3\to \mathbb{R}$, $h= h(x,y,z)$, um campo escalar definido em todos os pontos de uma superfície regular $S$. Assumimos que $\vec{r}(u,v)=x(u,v)\vec{i}+y(u,v)\vec{j}+z(u,v)\vec{k}$, $(u,v)\in R \subset \mathbb{R}^2$ seja uma parametrização para $S$. A integral de superfície de $h$ sobre $S$ é definida por:
\begin{equation}\label{definicao_int_sup_esc}
\iint_S h d S = \iint_R h(x(u,v),y(u,v),z(u,v)) \|\vec{r}_u\times \vec{r}_v\|d udv, 
\end{equation}
onde $dS$ é o elemento infinitesimal de área sobre a superfície.


\begin{ex}\label{ex_esfera_int_sup}Considere a esfera de raio $a$ 
$$
\vec{r}=a\sin(u)\cos(v)\vec{i}+a\sin(u)\sin(v)\vec{j}+a\cos(u)\vec{k},~~ a>0, ~ 0\leq u< \pi, ~ 0\leq v< 2\pi
$$
e o campo $h(x,y,z)=1$. Vamos calcular a área da esfera dada por $\iint_S h d S $. Começamos com as derivadas parciais de $\vec{r}$, dadas por
$$
\vec{r}_u=a\cos(u)\cos(v)\vec{i}+a\cos(u)\sin(v)\vec{j}-a\sin(u)\vec{k}
$$
e
$$
\vec{r}_v=-a\sin(u)\sin(v)\vec{i}+a\sin(u)\cos(v)\vec{j}.
$$
Então, temos
\begin{eqnarray}
\label{prod_vet_ru_rv}  \vec{r}_u\times\vec{r}_v&=&\left|
 \begin{array}{ccc}
 \vec{i} & \vec{j} & \vec{k} \\
  a\cos(u)\cos(v)&a\cos(u)\sin(v)&-a\sin(u) \\
-a\sin(u)\sin(v)&a\sin(u)\cos(v)&0
 \end{array}
\right|\\
\nonumber &=& a^2\sin^2(u)\cos(v)\vec{i}+ a^2\sin^2(u)\sin(v)\vec{j}\\\nonumber&+&(a^2\cos(u)\sin(u)\cos^2(v)+ a^2\cos(u)\sin(u)\sin^2(v))\vec{k}\\
\nonumber &=& a^2\sin^2(u)\cos(v)\vec{i}+ a^2\sin^2(u)\sin(v)\vec{j}+a^2\cos(u)\sin(u)\vec{k}.
\end{eqnarray}
e
\begin{eqnarray*}
\|  \vec{r}_u\times\vec{r}_v\|&=&a^2\sqrt{ \sin^4(u)(\cos^2(v)+ \sin^2(v))+\cos^2(u)\sin^2(u)}\\
&=&a^2\sqrt{ \sin^4(u)+\cos^2(u)\sin^2(u)}\\
&=&a^2\sin(u)\sqrt{ \sin^2(u)+\cos^2(u)}\\
&=&a^2\sin(u).
\end{eqnarray*}
onde usamos que $0\leq \sin(u)\leq 1$, visto que $0\leq u\leq \pi$. Portanto,
\begin{eqnarray*}
\iint_S h d S &=& \iint_R h(x(u,v),y(u,v),z(u,v)) \|\vec{r}_u\times \vec{r}_v\|d udv \\
&=& \int_0^{2\pi}\int_0^{\pi} a^2\sin(u) d udv \\
&=&2\pi a^2 \left[ -\cos(u) \right]_0^{\pi}\\
&=&4\pi a^2.
\end{eqnarray*}

\end{ex}
\subsection{Superfície definida como função de duas variáveis}
Nessa seção, vamos calcular a versão particular da definição \eqref{definicao_int_sup_esc} quando a superfície é definida como função de duas variáveis, isto é, $z=f(x,y)$ ou $y=f(x,z)$ ou ainda $x=f(y,z)$. Considere o caso onde a superfície $S$ é dado pela função $f:D\subset\mathbb{R}$, $D\subset \mathbb{R}^2$, $z=f(x,y)$ (os outros dois casos são análogos). Uma parametrização para a superfície $S$ é
$$
\vec{r}(x,y)= x\vec{i}+ y\vec{j}+ f(x,y)\vec{k},
$$
Calculamos as derivadas $\vec{r}_x=\vec{i}+ f_x\vec{k},$ e $\vec{r}_y=\vec{j}+ f_y\vec{k}$ e fazemos
$$
  \vec{r}_x\times\vec{r}_y=\left|
 \begin{array}{ccc}
 \vec{i} & \vec{j} & \vec{k} \\
  1 &0 & f_x \\
0 &  1 & f_y
 \end{array}
\right|\\= f_x\vec{i}+ f_y\vec{j}+\vec{k}.
$$
Agora, definimos $G$ tal que a superfície seja reescrita como $G(x,y,z)=0$, isto é, $G(x,y,z)=z-f(x,y)$ e observamos que
\begin{equation}\label{normal_grad_G_0}
   \vec{r}_x\times\vec{r}_y=\vec{ \nabla} G.
\end{equation}
Analogamente, em qualquer dos outros dois casos $y=f(x,z)$ ou $x=f(y,z)$, definimos $G$ tal que a superfície seja reescrita como $G(x,y,z)=0$ e a expressão \eqref{normal_grad_G_0} continua válida, isto é,
\begin{equation*}\label{normal_grad_G_0_2}
   \vec{r}_x\times\vec{r}_z=\vec{ \nabla} G\qquad \text{ou} \qquad    \vec{r}_y\times\vec{r}_z=\vec{ \nabla} G.
\end{equation*}
Portanto, a versão da definição \eqref{definicao_int_sup_esc} para o caso onde a superfície é definida por uma função de duas variáveis $f$ é dada por
\begin{equation}\label{definicao_int_sup_esc_par}
\iint_S h d S=  \iint_R h\| \vec{\nabla}G\| dA,
\end{equation}
onde $R$ é o domínio de $h$ e $h$ deve estar sobre os pontos da superfície.

\begin{ex}Vamos calcular novamente a integral de supefície do exemplo \ref{ex_esfera_int_sup}. Considere a equação da esfera de raio $a$, $x^2+y^2+z^2=a^2$ e o campo $h(x,y,z)=1$. Vamos calcular a área da superfície da esfera dada por $\iint_S h d S $. Para aplicar a expressão \eqref{definicao_int_sup_esc_par}, separamos a esfera em duas superfícies: $S_1$ de equação $z=h_1(x,y)=\sqrt{a^2-x^2-y^2}$ e $S_2$ de equação $z=h_2(x,y)=-\sqrt{a^2-x^2-y^2}$. Assim,
\begin{equation*}
\iint_S h d S=  \iint_{S_1} h d S+\iint_{S_2} h d S.
\end{equation*}
Definimos $G_1=z-\sqrt{a^2-x^2-y^2}$ e $G_2=z+\sqrt{a^2-x^2-y^2}$ e calculamos
$$
\vec{\nabla}G_1=\frac{x}{\sqrt{a^2-x^2-y^2}}\vec{i}+\frac{y}{\sqrt{a^2-x^2-y^2}}\vec{j}+\vec{k}
$$
e 
$$
\vec{\nabla}G_2=-\frac{x}{\sqrt{a^2-x^2-y^2}}\vec{i}-\frac{y}{\sqrt{a^2-x^2-y^2}}\vec{j}+\vec{k}
$$
Logo,
$$
\|\vec{\nabla}G_1\|=\|\vec{\nabla}G_2\|=\sqrt{\frac{x^2+y^2}{a^2-x^2-y^2}+1}=\sqrt{\frac{x^2+y^2+a^2-x^2-y^2}{a^2-x^2-y^2}}=\frac{a}{\sqrt{a^2-x^2-y^2}}.
$$
Como $h(x,y,z)=1$, temos:
\begin{equation*}
\iint_S h d S=  \iint_{D} \frac{a}{\sqrt{a^2-x^2-y^2}}d A+\iint_{D}\frac{a}{\sqrt{a^2-x^2-y^2}}dA=2\iint_{D} \frac{a}{\sqrt{a^2-x^2-y^2}}dA,
\end{equation*}
onde $D$ é o disco de raio $a$ no plano $xy$. Portanto, integrando em coordenadas polares, temos:
\begin{eqnarray*}
\iint_S h d S&=& 2 \iint_{D} \frac{a}{\sqrt{a^2-x^2-y^2}} dydx\\
&=& 2 \int_0^{2\pi}\int_{0}^a \frac{a}{\sqrt{a^2-r^2}} rdrd\theta\\
&=& 2\pi a \int_{0}^a \frac{2r}{\sqrt{a^2-r^2}} dr\\
&=& 2\pi a \left[\frac{-\sqrt{a^2-r^2}}{1/2}\right]_{0}^a\\
&=& 4\pi a^2.
\end{eqnarray*}

\end{ex}

\section{Definição de integral de superfície para um campo vetorial}
Considere $S$ uma superfície orientável e $\vec{r}(u,v)=x(u,v)\vec{i}+y(u,v)\vec{j}+z(u,v)\vec{k}$, $(u,v)\in R \subset \mathbb{R}^2$ uma parametrização regular, sendo $\vec{n}=\frac{\vec{r}_u\times \vec{r}_v}{\|\vec{r}_u\times \vec{r}_v\|}$ o vetor normal à $S$. Seja $\vec{F}:\mathbb{R}^3\to \mathbb{R}^3$, $\vec{F}=\vec{F}(x,y,z)$ um campo vetorial definido em todos em pontos de $S$. Então definimos a integral de superfície do campo vetorial $\vec{F}$ sobre $S$ como:
\begin{equation}\label{definicao_int_sup_vet}
\iint_S \vec{F}\cdot \vec{n} d S= \iint_R \vec{F}(x(u,v),y(u,v),z(u,v))\cdot (\vec{r}_u\times \vec{r}_v) dudv.
\end{equation}
\begin{ex}\label{ex_esfera_int_sup_vet}Seja $S$ a esfera de raio $a$ parametrizada por 
$$
\vec{r}=a\sin(u)\cos(v)\vec{i}+a\sin(u)\sin(v)\vec{j}+a\cos(u)\vec{k},~~ a>0, ~ 0\leq u< \pi, ~ 0\leq v< 2\pi
$$
e o campo $\vec{F}(x,y,z)=x\vec{i}+y\vec{j}+z\vec{k}$. Vamos calcular o fluxo do campo $\vec{F}$ através da superfície $S$ dado por $\Phi=\iint_S \vec{F}\cdot\vec{n} d S $. O produto vetorial
$$
\vec{r}_u\times\vec{r}_v= a^2\sin^2(u)\cos(v)\vec{i}+ a^2\sin^2(u)\sin(v)\vec{j}+a^2\cos(u)\sin(u)\vec{k}
$$
foi calculado na equação \eqref{prod_vet_ru_rv}. Observe que a parametrização define uma orientação para fora da superfície. De fato, basta pegar um ponto, por exemplo, $u=v=\pi/2$ e calcular $\vec{r}=a\vec{j}$ e $\vec{r}_u\times\vec{r}_v=a^2\vec{j}$, ou seja, no ponto $(0,a,0)$, a normal aponta para fora. Como
\begin{eqnarray*}
\vec{F}(x(u,v),y(u,v),z(u,v))&=&x(u,v)\vec{i}+y(u,v)\vec{j}+z(u,v)\vec{k}\\
&=&a\sin(u)\cos(v)\vec{i}+a\sin(u)\sin(v)\vec{j}+a\cos(u)\vec{k},
\end{eqnarray*}
temos
\begin{eqnarray*}
 \vec{F}\cdot (\vec{r}_u\times\vec{r}_v)&=& a^3\sin^3(u)\cos^2(v)+a^3\sin^3(u)\sin^2(v)+a^3\cos^2(u)\sin(u)\\
 &=& a^3\sin^3(u)+a^3\cos^2(u)\sin(u)\\
 &=& a^3\sin(u).
\end{eqnarray*}
Portanto,
\begin{eqnarray*}
 \Phi&=&\iint_S \vec{F}\cdot\vec{n} d S \\
&=& \iint_R \vec{F}\cdot (\vec{r}_u\times\vec{r}_v) d udv \\
&=& \int_0^{2\pi}\int_0^\pi a^3\sin(u) d udv \\
&=& 2\pi a^3\int_0^\pi \sin(u) d u \\
&=&4\pi a^3.
\end{eqnarray*}
\end{ex}
\subsection{Superfície definida como função de duas variáveis}
Nessa seção, vamos calcular a versão particular da definição \eqref{definicao_int_sup_vet} quando a superfície é definida como função de duas variáveis, isto é, $z=f(x,y)$ ou $y=f(x,z)$ ou ainda $x=f(y,z)$. Considere o caso onde a superfície $S$ é dado pela função $f:D\subset\mathbb{R}$, $D\subset \mathbb{R}^2$, $z=f(x,y)$ (os outros dois casos são análogos). Uma parametrização para a superfície $S$ é
$$
\vec{r}(x,y)= x\vec{i}+ y\vec{j}+ f(x,y)\vec{k},
$$
onde o vetor normal a superfície é
$$
\vec{n}=\pm\frac{\vec{r}_x\times \vec{r}_y}{\|\vec{r}_x\times \vec{r}_y\|}.
$$
Aqui, o sinal $\pm$ é escolhido para ajustar a orientação da parametrização à orientação da superfície definida por $f$. Calculamos as derivadas $\vec{r}_x=\vec{i}+ f_x\vec{k},$ e $\vec{r}_y=\vec{j}+ f_y\vec{k}$ e fazemos
$$
  \vec{r}_x\times\vec{r}_y=\left|
 \begin{array}{ccc}
 \vec{i} & \vec{j} & \vec{k} \\
  1 &0 & f_x \\
0 &  1 & f_y
 \end{array}
\right|\\= f_x\vec{i}+ f_y\vec{j}+\vec{k}.
$$
Agora, definimos $G$ tal que a superfície seja reescrita como $G(x,y,z)=0$, isto é, $G(x,y,z)=z-f(x,y)$ e observamos que
\begin{equation}\label{normal_grad_G}
   \vec{r}_x\times\vec{r}_y=\vec{ \nabla} G.
\end{equation}
Analogamente, em qualquer dos outros dois casos $y=f(x,z)$ ou $x=f(y,z)$, definimos $G$ tal que a superfície seja reescrita como $G(x,y,z)=0$ e a expressão \eqref{normal_grad_G} continua válida, isto é,
\begin{equation*}\label{normal_grad_G_2}
   \vec{r}_x\times\vec{r}_z=\vec{ \nabla} G\qquad \text{ou} \qquad    \vec{r}_y\times\vec{r}_z=\vec{ \nabla} G.
\end{equation*}
Portanto, a versão da definição \eqref{definicao_int_sup_vet} para o caso onde a superfície é definida por uma função de duas variáveis $f$ é dada por
\begin{equation}\label{definicao_int_sup_vet_par}
\iint_S \vec{F}\cdot \vec{n} d S= \pm \iint_R \vec{F}\cdot \vec{\nabla}G dA,
\end{equation}
onde $R$ é o domínio de $f$, $\vec{F}$ deve estar sobre os pontos da superfície e o sinal $\pm$ deve ser escolhido para que $\pm \vec{\nabla}G$ esteja no sentido da orientação da superfície.


\begin{ex}\label{ex_esfera_int_sup_vet_2}Nesse exemplo, vamos recalcular o fluxo do exemplo \ref{ex_esfera_int_sup_vet}. Seja $S$ a esfera de raio $a$ dada por 
$$
x^2+y^2+z^2=a^2,
$$
orientada para fora, e o campo $\vec{F}(x,y,z)=x\vec{i}+y\vec{j}+z\vec{k}$. Vamos calcular o fluxo do campo $\vec{F}$ através da superfície $S$ dado por $\Phi=\iint_S \vec{F}\cdot\vec{n} d S $. Para aplicar a expressão \eqref{definicao_int_sup_vet_par}, separamos a esfera em duas superfícies: $S_1$ de equação $z=h_1(x,y)=\sqrt{a^2-x^2-y^2}$ e $S_2$ de equação $z=h_2(x,y)=-\sqrt{a^2-x^2-y^2}$. Assim,
\begin{equation*}
\iint_S \vec{F}\cdot\vec{n} d S=  \iint_{S_1} \vec{F}_1\cdot\vec{n} d S+\iint_{S_2} \vec{F}_2\cdot\vec{n} d S.
\end{equation*}
Definimos $G_1=z-\sqrt{a^2-x^2-y^2}$ e $G_2=z+\sqrt{a^2-x^2-y^2}$ e calculamos
$$
\vec{\nabla}G_1=\frac{x}{\sqrt{a^2-x^2-y^2}}\vec{i}+\frac{y}{\sqrt{a^2-x^2-y^2}}\vec{j}+\vec{k}
$$
e 
$$
\vec{\nabla}G_2=-\frac{x}{\sqrt{a^2-x^2-y^2}}\vec{i}-\frac{y}{\sqrt{a^2-x^2-y^2}}\vec{j}+\vec{k}.
$$
Observe que $\vec{\nabla}G_1$ está no sentido para fora da superfície e $\vec{\nabla}G_2$ no sentido para dentro. Logo, vamos escolher um sinal positivo e outro negativo na definição \eqref{definicao_int_sup_vet_par}.
Assim, sabendo que o campo na superfície $S_1$ é dado por $\vec{F}_1=x\vec{i}+y\vec{j}+\sqrt{a^2-x^2-y^2}\vec{k}$ e, o campos na supefície $S_2$ é dado por $\vec{F}_2=x\vec{i}+y\vec{j}-\sqrt{a^2-x^2-y^2}\vec{k}$, temos:
\begin{eqnarray*}
\iint_S \vec{F}\cdot\vec{n} d S&=&  \iint_{D} \vec{F}_1\cdot \vec{\nabla}G_1 d A-\iint_{D} \vec{F}_2\cdot\vec{\nabla}G_2 d A\\
&=&  \iint_{D} \frac{x^2}{\sqrt{a^2-x^2-y^2}}+\frac{y^2}{\sqrt{a^2-x^2-y^2}}+\sqrt{a^2-x^2-y^2} d A\\&-&\iint_{D} -\frac{x^2}{\sqrt{a^2-x^2-y^2}}-\frac{y^2}{\sqrt{a^2-x^2-y^2}}-\sqrt{a^2-x^2-y^2} d A\\
&=&  2\iint_{D} \frac{x^2+y^2}{\sqrt{a^2-x^2-y^2}}+\sqrt{a^2-x^2-y^2} d A\\
&=&  2\iint_{D} \frac{a^2}{\sqrt{a^2-x^2-y^2}} d A,
\end{eqnarray*}
onde $D$ é o disco de raio $a$ no plano $z=0$. Portanto,
\begin{eqnarray*}
\Phi&=& \iint_S \vec{F}\cdot\vec{n} d S\\
&=&  2\int_0^{2\pi }\int_{0}^a\frac{a^2}{\sqrt{a^2-r^2}} rdrd\theta\\
&=&  2\pi a^2\left[ -\frac{\sqrt{a^2-r^2}}{1/2} \right]_{0}^a\\
&=&  4\pi a^3. 
\end{eqnarray*}


\end{ex}

\begin{ex}\label{int_sup_ex_1} Considere a superfície $S$
$$z=0,~ {x^2+y^2}\leq 1,$$
orientada no sentido decrescente do eixo $z$, isto é, $\vec{n}=-\vec{k}$. Considere também o campo vetorial dado por $\vec{F}=(2+z^2+x)\vec{k}$. Vamos calcular o valor do fluxo $\Phi=\iint_S\vec{F}\cdot \vec{n} dS$.

Como $z=f(x,y)=0,~ {x^2+y^2}\leq 1$, definimos a função $G(x,y,z) = z$ e calculamos
$$
\vec{\nabla} G=\vec{k}.
$$
Observe que $\vec{\nabla} G$ e $\vec{n}$ estão em sentidos oposto e, portanto, na aplicação da expressão \eqref{definicao_int_sup_vet_par}, vamos escolher o sinal negativo. Aplicamos o campo $\vec{F}=(2+z^2+x)\vec{k}$ sobre a superfície $z=0$ e obtemos $\vec{F}=(2+x)\vec{k}$. Assim,
\begin{eqnarray*}
\Phi&:=&\iint_{S} \vec{F}\cdot \vec{n}dS\\
&=&-\iint_A \vec{F}\cdot \vec{\nabla} G dA\\
&=&-\iint_A \left(2+x\right) dA
\end{eqnarray*}
Resolvemos em coordenadas polares da seguinte forma:
\begin{eqnarray*}
\Phi&=&-\int_0^1\int_0^{2\pi} \left(2+r\cos(\theta)\right) rd\theta dr\\
&=&-\int_0^1\int_0^{2\pi} \left(2r+r^2\cos(\theta)\right) d\theta dr\\
&=&-2\pi\int_0^{1} 2r dr-\int_0^1\left(\left[r^2\sin(\theta)\right]_{\theta=0}^{\theta=2\pi} \right)dr \\
&=&-2\pi.
\end{eqnarray*}
\end{ex}

\begin{ex}\label{int_sup_ex_2} Considere a superfície $S$
 $$z=f(x,y)=1- \sqrt{x^2+y^2},$$
orientada na direção crescente do eixo $z$, isto é, o vetor $\vec{n}$ tem componente na direção $\vec{k}$ sempre positiva. Considere também o campo vetorial dado por
$\vec{F}=(2+z^2+x)\vec{k}$. Vamos calcular o valor do fluxo $\Phi=\iint_S\vec{F}\cdot \vec{n} dS$.

Como $z=f(x,y)=1- \sqrt{x^2+y^2}$, definimos $G(x,y,z) = z-1+ \sqrt{x^2+y^2}$ e calculamos
$$
\vec{\nabla} G=\frac{x}{\sqrt{x^2+y^2}}\vec{i}+\frac{y}{\sqrt{x^2+y^2}}\vec{j}+\vec{k}.
$$
Observe que $\vec{\nabla} G$ e $\vec{n}$ estão no mesmo sentido e, portanto, na aplicação da expressão \eqref{definicao_int_sup_vet_par}, vamos escolher o sinal positivo. Aplicamos o campo $\vec{F}=(2+z^2+x)\vec{k}$ sobre a superfície $z=0$ e obtemos $\vec{F}=(2+(1-\sqrt{x^2+^2})^2+x)\vec{k}$. Assim,


\begin{eqnarray*}
\Phi&:=&\iint_{S} \vec{F}\cdot \vec{n}dS\\
&=&\iint_A \vec{F}\cdot \vec{\nabla} G dA\\
&=&\iint_A \left(2+\left(1- \sqrt{x^2+y^2}\right)^2+x\right) dA
\end{eqnarray*}
Resolvemos em coordenadas polares e obtemos
\begin{eqnarray*}
\Phi&=&\int_0^1\int_0^{2\pi} \left(2+(1- r)^2+r\cos(\theta)\right) rd\theta dr\\
&=&\int_0^1\int_0^{2\pi} \left(3r- 2r^2 + r^3+r^2\cos(\theta)\right) d\theta dr\\
&=&2\pi\int_0^{1}\left(3r- 2r^2 + r^3\right) dr+\int_0^1\left(\left[r^2\sin(\theta)\right]_{\theta=0}^{\theta=2\pi} \right)dr \\
&=&2\pi\left[3\frac{r^2}{2}- 2\frac{r^3}{3} + \frac{r^4}{4}\right]_0^1\\
&=&2\pi\left[\frac{3}{2}- \frac{2}{3} + \frac{1}{4}\right]_0^1=\frac{13\pi}{6}.
\end{eqnarray*}

\end{ex}



\section{O Teorema da Divergência de Gauss}
\begin{teo}
Seja $V$ o volume de um sólido cuja superfície $S$ é orientada para fora. Seja o campo vetorial $\vec{F}$ dado por
$$
\vec{F}=f(x,y,z)\vec{j}+g(x,y,z)\vec{i}+h(x,y,z)\vec{k},
$$
onde as funções $f$, $g$ e $h$ possuem todas as derivadas parciais de primeira ordem contínuas em algum conjunto aberto contendo $V$. Então:
$$
\oiint_S \vec{F}\cdot \vec{n} dS=\iiint_V \vec{\nabla}\cdot \vec{F} dV.
$$
\end{teo}

\begin{ex}\label{int_sup_ex_div_1} Considere a superfície fechada orientada para fora composta superiormente pela superfície de rotação descrita como
$$z=f(x,y)=1- \sqrt{x^2+y^2}$$ e inferiormente por
$$z=0,~ {x^2+y^2}\leq 1.$$
Seja o campo vetorial dado por
$\vec{F}=(2+z^2+x)\vec{k}$. Vamos calcular o valor do fluxo $\iint_S\vec{F}\cdot \vec{n} dS$ via teorema da divergência.

Temos $\vec{\nabla}\cdot \vec{F}=2z$. Assim,
\begin{eqnarray*}
\Phi&=&\iint_S \vec{F}\cdot \vec{n}dS\\
&=&\iiint_V \vec{\nabla}\cdot \vec{F} dV\\
&=&\int_0^1 \int_0^{2\pi}  \int_0^{1-r} 2z rdzd\theta dr\\
&=&\int_0^1 \int_0^{2\pi}   (1-r)^2 rd\theta dr\\
&=&2\pi \int_0^1    (1-r)^2 r dr\\
&=&2\pi \int_0^1    \left(r-2r^2+r^3\right) dr\\
&=&2\pi  \left[\frac{r^2}{2}-2\frac{r^3}{3}+\frac{r^4}{4}\right]_{0}^1\\
&=&2\pi  \left(\frac{1}{2}-\frac{2}{3}+\frac{1}{4}\right)=\frac{\pi}{6}.
\end{eqnarray*}

Observe que as superfícies dos exemplos \ref{int_sup_ex_1} e \ref{int_sup_ex_1} foram a superfície fechada do exemplo \ref{int_sup_ex_div_1}. De fato, o fluxo calculado naqueles dois exemplos via parametrização direta foram $-2\pi$ e $\frac{13\pi}{6}$, cuja soma é $\frac{\pi}{6}$, o mesmo valor calculado pelo Teorema da Divergência.
\end{ex}



\section{O Teorema de Stokes}

\begin{teo}
Seja $S$ uma superfície orientável, suave por partes, limitada por uma curva $C$, fechada, suave por partes e positivamente orientada com respeito a $S$. Seja o campo vetorial $\vec{F}$ dado por
$$
\vec{F}=f(x,y,z)\vec{j}+g(x,y,z)\vec{i}+h(x,y,z)\vec{k},
$$
onde as funções $f$, $g$ e $h$ possuem todas as derivadas parciais de primeira ordem contínuas em algum conjunto aberto contendo $S$. Então:
$$
\oint_C \vec{F}\cdot d \vec{r}=\iint_S \vec{\nabla}\times \vec{F}\cdot \vec{n} dS.
$$
\end{teo}
